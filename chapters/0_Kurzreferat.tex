% Kurzreferat:
\newpage
\chapter*{Kurzreferat}
\section*{Agile Metriken {-} Qualitätssicherung in agilen Teams}

Agile Vorgehensmodelle, insbesondere Scrum, sind bei modernen Entwicklungsteams sehr beliebt und weit verbreitet.
Ein Grundsatz von Scrum bildet der Empirismus, also die Theorie, dass Wissen aus Erfahrung erlangt wird.
Um ein solches Wissen, im Speziellen über Qualität, aufbauen zu können und basierend auf diesem Wissen zukünftige Entscheidungen treffen zu können, ist es hilfreich, bestimmte Qualitätsmerkmale in Form von Metriken bereitzustellen.
Qualität in agilen Teams lässt sich unterteilen in Produktqualität und Prozessqualität, deren Merkmale sich durch Metriken als Zahlenwerte darstellen lassen.
Solche Metriken sind pro Team individuell und müssen daher mit dafür geeigneten Methoden ermittelt werden.
Bei komplexeren Problemstellungen sind Standard-Metriken oft nicht mehr ausreichend und es müssen eigene, speziell auf das identifizierte Problem zugeschnittene, Metriken erstellt werden.
Die Metriken müssen dann in geeigneter Form dargestellt und dem Team zur Verfügung gestellt werden.
Die Ermittlung der Metriken erfolgte über das \ac{GQM}-Modell, einem Qualitätsmodell zur Bestimmung von Metriken, indem die Daten der vergangenen Retrospektiven ausgewertet wurden.
Zusätzlich wurden verschiedene Metriken in einer Umfrage vorgestellt und von den Teammitgliedern einzeln bewertet.
Dabei konnte gezeigt werden, dass die vom Team gewählten Metriken, bis auf wenige Ausnahmen den Metriken aus dem \ac{GQM}-Modell entsprachen.
Die zugrundeliegenden Daten für diese ermittelten Metriken werden täglich in den einzelnen Systemen entlang des Entwicklungsprozesses erzeugt und über Schnittstellen zur Verfügung gestellt.
Die Metriken werden mit einer selbst erstellten und quelloffen zur Verfügung gestellten Software erzeugt, in einem bereits vorhandenen Elastic-Stack gespeichert und in einem Dashboard dargestellt.
Auf dem Dashboard sind die Metriken visuell gruppiert, um allen Teammitgliedern eine möglichst effiziente Sicht bereitzustellen.
Zur Evaluierung wurde das System in einem relativ fortgeschrittenen Scrum Team über den Zeitraum von nicht ganz drei Sprints eingesetzt.
Dadurch konnte gezeigt werden, dass das zur Verfügung gestellte Werkzeug bereits nach kurzer Zeit von den Teammitgliedern genutzt wurde.
Auch wenn in dieser kurzen Zeit noch kein eindeutiger Qualitätsgewinn nachgewiesen werden konnte, wurde in den Interviews zur Evaluierung klar, dass der Zweck erkannt und das Dashboard bereits wie vorgesehen genutzt wurde.
Es gab sogar bereits Gedanken, gewisse Metriken zu kombinieren, um Korrelationen nachzuweisen und Qualitätseinbußen besser verstehen und vermeiden zu können, was genau dem Ziel dieser Arbeit entsprach.

% Abstract:
\newpage
\chapter*{Abstract}
\section*{Agile Metrics {-} Quality Assurance In Agile Teams}

Agile approaches, especially Scrum, are very popular among modern software development teams.
A fundamental principle of Scrum is empiricism, which is the theory of gaining new knowledge through experience.
To acquire such a knowledge, especially about quality, and make decisions in the future based on that knowledge, it is useful to publish certain quality characteristics.
Quality in agile teams can be subdivided into product and process quality, which characteristics can be presented as numeric values through metrics.
Such metrics are individual for each team and therefore need to be determinded choosing the proper methods.
For complex problems standard metrics are no more sufficient and custom metrics are needed, adapted for this kind of problem.
Then these metrics need to be published and presented to the team in a proper way.
The determination of the metrics was done with the \ac{GQM}-model, which is a quality model for examining metrics.
This was done by analyzing the data of the past retrospectives.
Also metrics were presented to the team in a survey and each team member had to rate those metrics.
It was shown that the metrics chosen by the team match the ones determined by the \ac{GQM}-model with only a few exceptions.
The data needed for those metrics is produced daily in the systems along the development process and provided by interfaces.
The metrics were generated by a self-constructed open-source software, saved in an Elastic Stack and published on a dashboard.
The dashboard was grouped visually to provide an optimal view on these metrics for every team member.
For evaluation purposes the system was tested in a matured scrum team for almost three sprints.
These tests showed that the tool provided to the team has been used by the members for the benefit of the whole team.
Even if there could not have been demonstrated a measurable quality improvement, the interviews showed that the purpose had become clear and the dashboard was used by the team members as planned.
There had even been some ideas about combining metrics to prove correlations between them and better understand and avoid loss in quality, which has exactly been the purpose of this work.
