% Kurzreferat:
\newpage
\chapter*{Kurzreferat}
\section*{Agile Metriken {-} Qualitätssicherung in agilen Teams}

Agile Vorgehensmodelle, insbesondere Scrum, sind heutzutage bei Entwicklungsteams sehr beliebt und weit verbreitet.
Eine Grundsäule von Scrum bildet der Empirismus, also die Theorie, dass Wissen aus Erfahrung erlangt wird.
Um ein solches Wissen, im Speziellen über Qualität, aufbauen zu können, um basierend auf diesem Wissen neue Entscheidungen treffen zu können, ist es hilfreich, bestimmte Qualitätsmerkmale in Form von Metriken bereitzustellen.
Qualität in agilen Teams lässt sich unterteilen in Produktqualität und Prozessqualität, deren Merkmale sich durch Metriken als Zahlenwerte darstellen lassen.
Solche Metriken sind pro Team individuell und müssen daher mit dafür geeigneten Methoden ermittelt werden.
Bei komplexeren Problemstellungen sind Standard-Metriken oft nicht mehr ausreichend und es müssen eigene, speziell auf das identifizierte Problem zugeschnittene, Metriken erstellt werden.
Die Metriken müssen dann in geeigneter Form dargestellt und dem Team zur Verfügung gestellt werden.
Die Ermittlung der Metriken erfolgte über das \ac{GQM}-Modell, einem Qualitätsmodell zur Bestimmung von Metriken.
Zusätzlich wurden verschiedenste Metriken in einer Umfrage vorgestellt und von den Teammitgliedern einzeln bewertet.
Dabei konnte gezeigt werden, dass die vom Team gewählten Metriken, bis auf wenige Ausnahmen den Metriken aus dem \ac{GQM}-Modell entsprachen.
Die zugrundeliegenden Daten für diese ermittelten Metriken werden täglich in den einzelnen Systemen entlang des Entwicklungsprozesses erzeugt und über Schnittstellen zur Verfügung gestellt.
Die Metriken wurden mit einer selbst erstellten und quelloffen zur Verfügung gestellten Software erzeugt, in einem bereits vorhandenen Elastic-Stack gespeichert und in einem Dashboard dargestellt.
Das Dashboard wurde visuell gruppiert, um allen Teammitgliedern eine möglichst effiziente Sicht auf die Metriken zu geben.
Zur Evaluierung wurde das System in einem relativ fortgeschrittenen Scrum Team über den Zeitraum von nicht ganz drei Sprints eingesetzt.
Dadurch konnte gezeigt werden, dass das zur Verfügung gestellte Werkzeug bereits nach kurzer Zeit von den Teammitgliedern genutzt wurde.
Auch wenn in dieser kurzen Zeit noch kein eindeutiger Qualitätsgewinn nachgewiesen werden konnte, wurde in den Interviews zur Evaluierung klar, dass der Zweck erkannt und das Dashboard bereits wie vorgesehen genutzt wurde.
Es gab sogar bereits Gedanken, gewisse Metriken zu kombinieren, um Korellationen nachweisen und Qualitätseinbußen besser verstehen und vermeiden zu können, was genau dem Ziel dieser Arbeit entsprach.

% Abstract:
\newpage
\chapter*{Abstract}
\section*{Agile Metrics {-} Quality Assurance In Agile Teams}

\begin{enumerate}
    \item Background / introduction / situation
    \item Present research / purpose
    \item Methods / materials / subjects / procedures
    \item Results / findings
    \item Discussion / conclusion / implications / recommendations
\end{enumerate}
