% Kurzreferat:
\newpage
\chapter*{Kurzreferat}
\section*{Agile Metriken {-} Qualitätssicherung in agilen Teams}

Agile Vorgehensmodelle, insbesondere Scrum, sind heutzutage sehr beliebt und weit verbreitet.
Eine Grundsäule von Scrum bildet der Empirismus, also die Theorie, dass Wissen aus Erfahrung erlangt wird.
Um ein solches Wissen, im Speziellen über Qualität, aufbauen zu können, um basierend auf diesem Wissen neue Entscheidungen treffen zu können, ist es hilfreich, bestimmte Qualitätsmerkmale in Form von Metriken bereitzustellen.
\\
Diese Metriken sind pro Team individuell und müssen daher mit dafür geeigneten Methoden ermittelt werden.
Bei komplexeren Problemstellungen sind Standard-Metriken oft nicht mehr ausreichend und es müssen eigene, speziell auf das identifizierte Modell zugeschnittene, Metriken erstellt werden.
Diese Metriken müssen dann in geeigneter Form dargestellt und dem Team zur Verfügung gestellt werden.
\\


% Abstract:
\newpage
\chapter*{Abstract}
\section*{Agile Metrics {-} Quality Assurance In Agile Teams}

\begin{enumerate}
    \item Background / introduction / situation
    \item Present research / purpose
    \item Methods / materials / subjects / procedures
    \item Results / findings
    \item Discussion / conclusion / implications / recommendations
\end{enumerate}
