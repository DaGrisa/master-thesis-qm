\chapter{Einleitung}

Dieses Kapitel gibt eine kurze Übersicht über den Aufbau und die Motivation hinter dieser Arbeit.

\section{Zielsetzung}

Agile Prozesse, insbesondere Scrum, basieren auf empirischen, also durch fortlaufende Beobachtung erhobenen und auswertenden, Daten (siehe Abschnitt~\ref{section:scrum}).
Das bedeutet, dass sich der Prozess durch Reflexion verbessert und auch Veränderung zulässt.
Um diese Reflexion zu vereinfachen, ist es hilfreich, gewisse Kennzahlen des Prozesses und des Produktes grafisch darzustellen.
Im Speziellen Metriken können eine gute qualitative Auskunft über den aktuellen Status geben.
Ziel dieser Arbeit soll es sein, Metriken zu ermitteln, die Qualitätsprobleme im Entwicklungsprozess oder im Softwareprodukt quantitativ abbilden können.
Weiters sollen diese Metriken gesammelt und grafisch dargestellt werden.
Sie können aus Daten generiert werden, die bei der tagtäglichen Arbeit in den jeweiligen Systemen erzeugt werden.
Solche Systeme reichen von \acfp{VCS}, über \acfp{PTS} und \acf{CI} / \acf{CD}, bis hin zu \acf{APM}.
Diese Daten können meist über Schnittstellen abgefragt und anschließend aggregiert abgelegt werden.
Umgesetzt wird das Ganze in einem relativ jungen, aber im Scrum Prozess bereits weit fortgeschrittenen Scrum-Team.

\clearpage
\section{Aufbau der Arbeit}

Nach der Einleitung gibt das Kapitel ``Stand der Technik'' einen Einblick in die unterschiedlichen Themen und die theoretischen Hintergründe dieser Arbeit.
Zuerst werden die Grundsäulen der agilen Softwareentwicklung, das agile Manifest und die agilen Prinzipien, genauer beschrieben.
Darauf folgend wird auf Scrum eingegangen, zusätzlich werden Ansätze für Scrum in mehreren Teams aufgezeigt.
Anschließend wird zu Qualität übergegangen, im Speziellen Software- und Prozessqualität.
Basierend auf den beiden vorherigen Themen werden Metriken genauer erörtert, was auch der Hauptteil dieses Kapitels darstellt.
Im ersten Teil werden Metriken aus den unterschiedlichsten Systemen vorgestellt und die Erstellung eigener Metriken erläutert.
Danach folgen Hinweise zur Veröffentlichung von Metriken und der Messung von Agilen Prinzipien.
Den Schluss bildet noch ein Überblick über Qualitätsmodelle und das \ac{FCM}-Modell im Detail, sowie die \ac{GQM}-Methodik.
\\
Im Kapitel ``Vorgehensweise'' wird beschrieben, wie bei der Bestimmung der relevanten Metriken für das Team, bei der Erstellung der Software und bei der Evaluierung der Ergebnisse vorgegangen wird.
\\
Das Kapitel ``Umsetzung'' soll in weiterer Folge, wie der Titel schon sagt, die Umsetzung der Lösung zeigen.
Anfangs werden die Gegebenheiten erläutert, in denen die erstellte Software eingesetzt wird.
Weiters wird mit der Identifizierung der Metriken gestartet und diese anschließend gesammelt.
Zuletzt wird die Darstellung detaillierter aufgezeigt.
\\
Es folgt das Kapitel ``Evaluierung'', in dem die Ergebnisse qualitativ in Form von Interviews und quantitativ in Form der Metriken evaluiert werden.
In den Kapiteln ``Schlussfolgerungen'' und ``Zusammenfassung'' werden die Ergebnisse nochmal reflektiert, zusammengefasst und ein Ausblick für mögliche weitere Arbeiten gegeben.
