\chapter{Einleitung}

(Mit Qualitäts-Analysetools, wie z.B. SonarQube\footcite{sonarqube}, können ganze Softwaresysteme kontinuierlichen Qualitätstests unterzogen werden.
Durch die ermittelten Kennzahlen können Aussagen zur Qualität des gesamten Systems, über einzelne Komponenten, bis hin zu einer einzelnen Quellcode-Datei getroffen werden.

Auch Scrum\footcite{scrum} wird als agiles Vorgehensmodell in der Softwareentwicklung immer beliebter. 
Dabei wird bei mehreren Teams, die auf vielen Systeme arbeiten, auf 2 Arten von Scrum Teams zurückgegriffen:
Feature- oder Komponenten-Teams.

Diese Arbeit beschäftigt sich mit dem Qualitätsmanagement in Scrum-Teams.
Das bedeutet, dass Kennzahlen zu Qualitätsmerkmalen nicht auf System-, sondern auf Komponentenebene gesammelt und aggregiert werden, um für jedes Team eine individuelle Sicht auf das Qualitätsmanagement bereitzustellen.
Um das zu ermöglichen, wird erst eine Vorgehensweise zur Ermittlung von relevanten Kennzahlen entwickelt und diese an einer Beispiel-Organisation angewendet.
Zur Sammlung, Auswertung und Darstellung dieser Kennzahlen wird eine Software entwickelt, die in eine bestehende Umgebung integriert werden kann.)

\ldots schreibe ich ganz am Schluss neu
