\chapter{Einleitung}

Dieses Kapitel gibt eine kurze Übersicht über den Aufbau und die Motivation hinter dieser Arbeit.

\section{Zielsetzung}

Agile Prozesse, insbesondere Scrum, basieren auf Empirismus (siehe Abschnitt~\ref{section:scrum}).
Das bedeutet, dass sich der Prozess durch Reflexion verbessert und auch Veränderung zulässt.
Um diese Reflexion zu vereinfachen, ist es hilfreich, gewisse Kennzahlen des Prozesses und des Produkts grafisch darzustellen.
Im Speziellen Metriken können eine gute qualitative Auskunft über den aktuellen Status geben.
Ziel dieser Arbeit soll es sein, Metriken zu ermitteln, die qualitative Schwachstellen im Entwicklungsprozess oder im Softwareprodukt abbilden können.
Weiters sollen diese Metriken gesammelt und grafisch dargestellt werden.
Sie können aus Daten generiert werden, die bei der tagtäglichen Arbeit in den jeweiligen Systemen erzeugt werden.
Solche Systeme reichen von \acfp{VCS}, über \acfp{PTS} und \acf{CI} / \acf{CD}, bis hin zu \acf{APM}.
Diese Daten können meist über Schnittstellen abgefragt und anschließend aggregiert abgelegt werden.
Umgesetzt wird das Ganze in einem relativ jungen, aber im Scrum Prozess bereits weit fortgeschrittenen Scrum-Team.

\clearpage
\section{Aufbau der Arbeit}

Einen Einblick in die unterschiedlichen Themen und die theoretischen Hintergründe dieser Arbeit gibt das Kapitel ``Stand der Technik''.
Zuerst werden die Grundsäulen der agilen Softwareentwicklung, das agile Manifest und die agilen Prinzipien, genauer erklärt.
Darauf folgend wird Scrum genauer erklärt, zusätzlich Ansätze für Scrum in mehreren Teams.
Anschließend wird zu Qualität übergegangen, im Speziellen Software- und Prozessqualität.
Basierend auf den beiden vorherigen Themen, wird dann genauer auf Metriken eingegangen, was auch der Hauptteil dieses Kapitels darstellt.
Im ersten Teil werden Metriken aus den unterschiedlichsten Systemen vorgestellt und wie eigene Metriken erstellt werden können.
Danach folgen Hinweise zur Veröffentlichung von Metriken und der Messung von Agilen Prinzipien.
Am Schluss folgen noch ein Überblick über Qualitätsmodelle und das \ac{GQM}-Modell, sowie das \ac{FCM}-Modell im Detail.
\\
Im Kapitel ``Vorgehensweise'' wird beschrieben, wie bei der Bestimmung der relevanten Metriken für das Team, bei der Erstellung der Software und bei der Evaluierung der Ergebnisse vorgegangen wird.
\\
Das Kapitel ``Umsetzung'' zeigt dann, wie der Titel schon sagt, die Umsetzung der Lösung.
Anfangs werden die Gegebenheiten erläutert, in der die Lösung eingesetzt wird.
Dann wird mit der Identifizierung der Metriken gestartet und diese anschließend in der entwickelten Software gesammelt.
Zuletzt wird noch genauer auf die Darstellung eingegangen.
\\
Am Ende folgt das Kapitel ``Evaluierung'', in dem die Ergebnisse qualitativ in Form von Interviews und quantitativ in Form der Metriken evaluiert werden.
In den Kapiteln ``Schlussfolgerungen'' und ``Zusammenfassung'' werden die Ergebnisse nochmal reflektiert, zusammengefasst und ein Ausblick für mögliche weitere Arbeiten gegeben.
