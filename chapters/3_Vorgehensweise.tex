\chapter{Vorgehensweise}

Agile Methoden, im speziellen Scrum, sind heutzutage in der Softwareentwicklung sehr weit verbreitet.
Ein wichtiges Werkzeug dieser Methoden ist der evolutionäre Ansatz, der in Form von Retrospektiven (bei Scrum) zur kontinuierlichen Verbesserung des agilen Prozesses beitragen soll.
In diesen Retrospektiven werden dann auch Maßnahmen getroffen, um solche Verbesserungen umzusetzen.
In dieser Arbeit soll ein Vorgehensmodell entwickelt werden, wie solche Verbesserungen oder auch Defizite messbar und somit sichtbar gemacht werden können.
Weiters soll eine Software entwickelt werden, um die notwendigen Daten zu sammeln und darstellen zu können.

\section{Vorgehensmodell}

Entwicklung eines Vorgehensmodells zur Bestimmung von relevanten Qualitätsmetriken von agilen Teams.
Dabei müssen folgende Punkte beachtet werden:
\begin{itemize}[noitemsep]
    \item Ebene für die die Metriken bestimmt sind (agiles Team, mittleres Management, Geschäftsleitung)
    \item allgemeine Metriken für diese Ebene
    \item spezielle Probleme erkennen und Metriken dazu erstellen
\end{itemize}

\section{Software}

Entwicklung einer Software zum Sammeln von Kennzahlen zur Erstellung von Qualitätsmetriken. 
Dabei müssen folgende Kriterien beachtet werden:
\begin{itemize}[noitemsep]
    \item Umsetzung in Java
    \item einzubindende Systeme: BitBucket Server~\footcite{bitbucket_server}, JIRA~\footcite{jira}, Jenkins~\footcite{jenkins}, SonarQube~\footcite{sonarqube}, Icinga~\footcite{icinga}
    \item Speicherung und Darstellung der Metriken erfolgt in einem Elastic Stack~\footcite{elastic_stack}
\end{itemize}
