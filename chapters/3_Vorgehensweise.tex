\chapter{Vorgehensweise}

In diesem Kapitel wird erklärt, wie bei der Erarbeitung der Lösung vorgegangen wird.
Zuerst werden die benötigten Metriken ermittelt.
Dazu werden die Daten der letzten Retrospektiven analysiert und eine Umfrage im Team gemacht.
Diese Metriken werden dann in einer Software regelmäßig erzeugt, gespeichert und angezeigt.
Zuletzt wird das Ergebnis der Arbeit qualitativ und quantitativ evaluiert.

\section{Metriken bestimmen}

Das Unternehmen, in dem die Tests durchgeführt werden, arbeitet seit rund einem Jahr nach dem Scrum Framework.
Als Basis zur Bestimmung der Metriken können daher die vorhandenen Retrospektiven genutzt werden, da diese eine Richtung vorgeben, in die sich das Team bewegen will.
Es können die meistgenutzten Schlagwörter aus den Fragestellungen der Retrospektive ermittelt werden und mithilfe der \ac{GQM}-Methodik in Metriken abgebildet werden.
Das \ac{FCM}-Modell ist hier weniger geeignet, da die Teams in dem Unternehmen, in dem die Software getestet wird, nicht an einem Produkt arbeiten, sondern an Geschäftsprozessen.
Außerdem wollen sie den Scrum-Prozess verbessern, nicht nur ein Produkt.
\\
\\
Als Ergänzung zur \ac{GQM}-Methodik wird zusätzlich das Team in einer Umfrage zu möglichen Metriken befragt.
Dabei werden die in dieser Arbeit erarbeiteten Metriken genauer vorgestellt und von den Teammitgliedern bewertet.
Dadurch soll zusätzlich die Möglichkeit gegeben werden, Metriken oder Probleme, die zuvor nicht genannt wurden, angeben zu können.
Außerdem können die Ergebnisse der \ac{GQM}-Methodik mit denen der Umfrage verglichen werden.

\clearpage
\section{Software}

Im Rahmen dieser Arbeit wird eine Software entwickelt, die Qualitätsmetriken aus unterschiedlichen Systemen ermitteln und bereitstellen kann.
Dabei sollen die zuvor ermittelten Metriken automatisch erzeugt und gespeichert werden.

\subsection{Anforderungen}\label{vorgehen:software}

Die Anforderungen an die Software entstanden zum einen aus der gewünschten Funktionsweise und zum anderen aus den Gegebenheiten des Umfelds der Software (und dem Unternehmen, in dem sie getestet werden soll).

\begin{description}
    \item[Erweiterbarkeit] \hfill \\ Um einfach neue Systeme und Metriken bereitstellen zu können, muss bei der Architektur auf eine einfache Erweiterbarkeit geachtet werden.
    \item[Fehlertoleranz] \hfill \\ Ein Fehler in einem einzelnen System, das Daten bereitstellt, darf nicht zum Absturz der Software führen.
    \item[Qualitätssicherung] \hfill \\ Die Qualitätssicherung muss nach aktuellen Standards erfolgen, das bedeutet: Einzelne Komponenten der Software sollten eine möglichst hohe Testabdeckung aufweisen. \ac{CI} mit statischer Codeanalyse muss eingerichtet werden.
    \item[Dokumentation] \hfill \\ Es muss auf eine möglichst verständliche und vollständige Dokumentation geachtet werden, um die Möglichkeit der Weiterentwicklung offen zu lassen.
    \item[Umsetzung in Java] \hfill \\ Java ist in vielen Unternehmen verbreitet und stößt daher auf eine hohe Akzeptanz.
    \item[einzubindende Systeme] \hfill \\ Metriken können in BitBucket Server~\footcite{bitbucket_server}, JIRA~\footcite{jira}, Jenkins~\footcite{jenkins}, SonarQube~\footcite{sonarqube} oder Icinga~\footcite{icinga} vorkommen. Systeme, in denen relevante Metriken ermittelt wurden, müssen unterstützt werden.
    \item[Speicherung und Darstellung] \hfill \\ Speicherung und Darstellung der Metriken erfolgt in einem Elastic Stack~\footcite{elastic_stack}.
\end{description}

\clearpage
\section{Evaluierung}

Nach der Fertigstellung der Software wird diese für Testzwecke in einem Unternehmen für mehrere Sprints eingesetzt, um über einen möglichst langen Zeitraum Metriken zu sammeln.
Parallel dazu wird den betroffenen Teammitgliedern der Umgang mit dem Dashboard näher gebracht.
\\
Die Evaluierung wird zum Einen quantitativ durchgeführt, da sich Metriken sehr gut dafür eignen.
Allerdings kann es sein, dass die Zeit, die für den Test zur Verfügung steht, einfach zu kurz ist, um eine Tendenz in den Metriken erkennen zu können.
\\
Daher wird zum Anderen noch eine qualitative Evaluierung durchgeführt, in Form von Interviews mit ausgewählten Teammitgliedern.
Dies dient der Ermittlung der Effektivität der eingesetzten Methoden und der Erfragung des subjektiven Empfindens gegenüber der Nützlichkeit von Metriken im agilen Prozess.