\chapter{Evaluierung}

\section{Quantitative Evaluierung}

Anhand der zeitlichen Entwicklung der gemessenen Metriken wäre eine quantitative Evaluierung möglich.
Allerdings ist einerseits der Zeitraum von 2-3 Sprints noch zu gering, um eine Entwicklung auf die Maßnahmen rückschließen zu können,
andererseits fehlen bei manchen Metriken die Werte vor der Inbetriebnahme, was eine Änderung ab der Inbetriebnahme ebenfalls schwer erkennen lässt.

\clearpage
\section{Qualitative Evaluierung}

Zur qualitativen Evaluierung werden Interviews mit dem Scrum-Master, dem Product Owner und einem Entwickler geführt.
Das ermöglicht einen Einblick aus Managementsicht (Product Owner) und aus Prozesssicht (Scrum Master und Entwickler), welche womöglich ganz unterschiedliche Anforderungen an das Produkt stellen.
Dabei soll herausgefunden werden, ob die richtigen Metriken ermittelt wurden und welche Metriken als besonders nützlich angesehen werden.
Ebenfalls wird nach einer nachweisbaren und spürbaren Qualitätsverbesserung gefragt.
Interessant ist auch noch wann und wie das Dashboard genutzt wird und wie es um die Benutzbarkeit steht.

\subsection{Interview-Fragen}

Folgende Interviewfragen sollen helfen, den roten Faden im Gespräch zu behalten.

\begin{enumerate}
    \item War dir von Anfang an klar, was das Ziel dieses Projektes ist? War die Anforderung / Problemstellung klar?
    \item Wird das Dashboard von dir genutzt? Wenn ja, wann und wie nutzt du das Dashboard?
    \item Wie ist der Zugang und die Bedienbarkeit des Dashboards?
    \item Ist das Dashboard übersichtlich und klar eingeteilt?
    \item Rückblickend gesehen, wurden die richtigen Metriken ermittelt und auf dem Dashboard visualisiert?
    \item Welche Metriken auf dem Dashboard sind besonders wichtig oder werden oft genutzt?
    \item Ist bereits eine Qualitätsverbesserung im Prozess oder in einem Produkt spürbar? Oder sogar nachweisbar?
    \item Gibts es aus deiner Sicht Verbesserungspotential? Wo liegen aus deiner Sicht die Schwächen dieser Lösung?
\end{enumerate}

Die Interviews wurden mit Zustimmung der Interviewten aufgezeichnet, ein Transkript befindet sich in Anhang XYZ.
Interne und vertrauliche Informationen wurden mit *** ersetzt.

\subsection{Interview-Antworten}
