\chapter{Evaluierung}

Um die Effektivität der erarbeiteten Lösung feststellen zu können, wird eine Evaluierung durchgeführt.
Die dargestellten Metriken bieten sich an, um eine quantitative Evaluierung durchzuführen.
Zusätzlich dazu werden ausgewählte Teammitglieder interviewt, um eine qualitative Evaluierung durchführen zu können.

\section{Quantitative Evaluierung}

Anhand der zeitlichen Entwicklung der gemessenen Metriken wäre eine quantitative Evaluierung möglich.
Allerdings ist einerseits der Zeitraum von 2{-}3 Sprints noch zu gering, um eine Entwicklung auf die Maßnahmen rückschließen zu können,
andererseits fehlen bei manchen Metriken die Werte vor der Inbetriebnahme, was eine Änderung ab der Inbetriebnahme ebenfalls schwer erkennen lässt.

\clearpage
\section{Qualitative Evaluierung}

Zur qualitativen Evaluierung werden Interviews mit dem Scrum-Master, dem Product Owner und einer Entwicklerin geführt.
Das ermöglicht einen Einblick aus Managementsicht (Product Owner) und aus Prozesssicht (Scrum Master und Entwicklerin), welche womöglich ganz unterschiedliche Anforderungen an das Produkt stellen.
Dabei soll herausgefunden werden, ob die richtigen Metriken ermittelt wurden und welche Metriken als besonders nützlich angesehen werden.
Ebenfalls wird nach einer nachweisbaren und spürbaren Qualitätsverbesserung gefragt.
Interessant ist auch noch wann und wie das Dashboard genutzt wird und wie es um die Benutzbarkeit steht.

\subsection{Interview-Fragen}

Folgende Interviewfragen sollen helfen, den roten Faden im Gespräch zu behalten.
Als Einstieg wird das Dashboard geöffnet, um es während dem Gespräch immer sichtbar vor sich zu haben.

\begin{enumerate}
    \item Wird das Dashboard von dir genutzt? Wenn ja, wann und wie nutzt du das Dashboard?
    \item Wie ist der Zugang und die Bedienbarkeit des Dashboards?
    \item Ist das Dashboard übersichtlich und klar eingeteilt?
    \item Rückblickend gesehen, wurden die richtigen Metriken ermittelt und auf dem Dashboard visualisiert?
    \item Welche Metriken auf dem Dashboard sind besonders nützlich oder werden oft genutzt?
    \item Ist bereits eine Qualitätsverbesserung im Prozess oder in einem Produkt spürbar? Oder sogar nachweisbar?
    \item Gibts es aus deiner Sicht Verbesserungspotential? Wo liegen aus deiner Sicht die Schwächen dieser Lösung?
\end{enumerate}

Die Interviews wurden mit Zustimmung der Interviewten aufgezeichnet, ein Transkript befindet sich in Anhang~\ref{appendix:transcript}.
Interne und vertrauliche Informationen wurden mit *** ersetzt.

\clearpage
\subsection{Interview-Antworten}

Auffallend bei den Antworten sind die unterschiedlichen Sichten auf das Dashboard.
Während der Product Owner das Dashboard mehr für Werbezwecke genutzt hat, um andere Bereiche auf die Möglichkeit von Metriken aufmerksam zu machen, hat es es der Scrum Master eher für die Langzeit-Sicht des Teams genutzt und seinen Fokus auf die Verbesserung des Prozesses gelegt.
Die Entwicklerin wiederum hatte den Fokus auf die kurzfristigen Metriken, wie den Bug Count, um schnell auf Probleme reagieren zu können.
Genutzt wurde es daher von allen Befragten, wichtiger ist aber, dass alle einen Nutzen darin sahen und es für sich bestmöglich nutzen können.
Bei der Bedienbarkeit sind einzelne Schwächen aufgetaucht, vor allem was die Detailansicht von Kennzahlen betrifft.
Die Einteilung wurde von allen als klar gesehen, eine Kennzeichnung der betroffenen Rollen (Scrum Master, Product Owner oder Entwicklerinnen) wäre eventuell noch sinnvoll.
Auch eine genauere Beschreibung der Ziele und gewisser Metriken, könnte Missverständnisse verhindern.
Die Auswahl der Metriken wurde als gut empfunden, manche Metriken machen erst Sinn, wenn sie mit anderen kombiniert werden.
Diese Kombination von Metriken war aber auch vor allem dem Scrum Master klar und wird in absehbarer Zukunft auch noch so umgesetzt.
Welche Metriken als besonders nützlich angesehen werden, hängt stark von der Rolle ab.
Wie bereits anfangs erwähnt, interessiert sich die Entwicklerin mehr für die kurzzeitigen Metriken, während das Interesse des Scrum Masters und Product Owners mehr bei den Langzeit-Metriken liegt.
Eine Qualitätsverbesserung ist noch nicht klar ersichtlich, da es ein noch zu kurzer Zeitraum ist, um darüber Aussagen treffen zu können.
Was aber von der Entwicklerin als spürbar erwähnt wurde, ist der Umgang mit den Bugs, die nun frühzeitig erkennt und abgearbeitet werden können.
Verbesserungspotential sehen alle noch bei der Nutzung der Metriken, was aber auch vor allem auf den zu kurzen Zeitraum der Tests zurückschließen ist.
Einige gute Vorschläge, wie das Beschreiben von Zielen, eine bessere Kategorisierung nach Rollen und das Interesse an neuen Metriken zeigt aber, dass das Dashboard durchaus genutzt und das Ziel dahinter, das Ganze als Wekrzeug für das gesamte Team zu nutzen, bereits erkannt wurde.
