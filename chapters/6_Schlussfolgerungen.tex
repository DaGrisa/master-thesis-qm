\chapter{Schlussfolgerungen}

In dieser Arbeit konnte in einer Fallstudie gezeigt werden, dass mithilfe des \ac{GQM}-Modells, ergänzt durch eine Umfrage im entsprechenden Scrum-Team, Metriken ermittelt werden können, die es ermöglichen, die Schwachstellen in einem Produkt und im agilen Prozess in Zahlen zu fassen.
Diese Metriken ermöglichen es dem agilen Team, seine Fortschritte bei den Retrospektiven nachvollziehbar zu bewerten und weitere Maßnahmen zu treffen.
Außerdem werden durch die Kombination unterschiedlicher Metriken Korrelationen zwischen bestimmten Metriken nachgewiesen oder widerlegt.
Reichen die bekannten Metriken in der Literatur nicht aus, besteht die Möglichkeit, dass auch eigene Metriken erstellt werden.
\\
Die entwickelte Software hilft dabei, die Daten aus den unterschiedlichen Systemen im Entwicklungsprozess als Metriken aufzubereiten und zu speichern.
Dabei wurde bei der Architektur auf Fehlertoleranz und einfache Erweiterbarkeit geachtet, sodass ein einfacher Betrieb und eine Erweiterung der unterstützten Systeme und Metriken möglich ist.
Für einen einfachen Zugang und eine uneingeschränkte Erweiterbarkeit wurde der Quellcode der Software unter der quelloffenen MIT-Lizenz veröffentlicht.
Bei der Visualisierung von Metriken bietet Kibana eine geeignete Plattform, um aus den gespeicherten Metriken einfach Dashboards mit unterschiedlichen Visualisierungen bereitzustellen.
Dabei muss berücksichtigt werden, dass Entwicklerinnen, Scrum-Master und Product-Owner jeweils unterschiedliche Interessen an den Metriken haben.
Durch eine geeignete Gruppierung der Metriken ist es möglich, den unterschiedlichen Teammitgliedern auf einen Blick die wichtigsten Metriken anzuzeigen.
Dadurch wird das Dashboard auch regelmäßig genutzt und somit die Akzeptanz noch weiter erhöht.
Der Einsatz der Software ermöglicht den Aufbau einer zentralen Stelle, an der die Metriken aus allen relevanten System gesammelt, dargestellt und kombiniert werden.
\\
Schwachstellen wurden bei der Art der Darstellung mancher Metriken identifiziert.
So wurde zum Beispiel der Wunsch nach einer besseren Beschreibung der Metriken, um Diskussionen über deren Bedeutung zu vermeiden, geäußert.
Zusätzlich soll künftig eine detailliertere Beschreibung von Zielen helfen, die Absicht hinter Metriken zu erklären.
Bei manchen Darstellungen wurden noch zusätzliche Detailinformationen zu den einzelnen Datensätzen gewünscht, um Ausreißer leichter zu identifizieren.
Trotzdem wurde das erstellte Dashboard trotz des relativ kurzen Testzeitraums von nicht ganz drei Sprints gut angenommen.
Bei der Evaluierung wurde auch klar, dass die Teammitglieder bereits erkannten, wie sie das Dashboard persönlich am besten nutzen.
\\
Durch den Einsatz der entwickelten Software und der vorgestellten Modelle zur Identifizierung von relevanten Metriken, kann die Qualität in einem agilen Team dadurch erhöht werden, dass Qualitätsprobleme durch Metriken sichtbar gemacht und in den Retrospektiven Gegenmaßnahmen dafür getroffen werden können.
