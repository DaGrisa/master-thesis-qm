\chapter{Zusammenfassung}

Scrum ist inzwischen eine sehr weit verbreitete agile Vorgehensweise von Entwicklungsteams.
Dabei ist die Reflexion der bisherigen Arbeit eine der Grundideen von Scrum, der sogenannte Empirismus.
Empirismus ist die Theorie, dass Wissen aus Erfahrung erlangt wird.
Um auf Basis dieses Wissens Entscheidungen zu treffen, benötigt ein Scrum-Team Kennzahlen.
Solche Kennzahlen können in Form von Metriken in einem leicht zugänglichen Dashboard visualisiert werden.
Metriken dienen hauptsächlich dazu, Qualitätsmerkmale eines Produkts oder Prozesses als Zahlenwert darzustellen.
\\
Eingeführt wurde ein solches Dashboard bei einem relativ fortgeschrittenen Scrum-Team in einem Unternehmen mit rund 6700 Mitarbeitern.
Gestartet wurde mit der Ermittlung von geeigneten Metriken für das Team.
Dazu wurden zuerst alle bisherigen Retrospektiven ausgewertet und basierend auf diesen Daten mit dem \ac{GQM}-Modell Metriken ermittelt.
Zusätzlich wurde aufgrund des erfahrenen Teams noch eine Umfrage gemacht, bei der den Teammitgliedern die gängigsten Metriken vorgestellt und in einer Skala von eins bis zehn nach Wichtigkeit bewertet wurden.
\\
Um diese ermittelten Metriken dann darzustellen, wurde eine leicht erweiterbare, quelloffene Software in Java erstellt, die alle relevanten Daten aus den Systemen im Entwicklungsprozess über Schnittstellen sammelt, aufbereitet und als Metriken in einer Datenbank speichert.
Konkret wurde der sogenannte Elastic Stack eingesetzt, also eine ElasticSearch Datenbank zur Speicherung und Kibana zur Visualisierung der Metriken und Bereitstellung über Dashboards.
Diese Metriken wurden anschließend vom Scrum-Team über den Zeitraum von nicht ganz drei Sprints getestet.
Dabei konnte gezeigt werden, dass je nach Rolle im Team zwar andere Metriken wichtig sind, aber jeder das Dashboard für sich zu nutzen wusste.
Besonders der Scrum-Master bekam schnell einen Eindruck, welche Möglichkeiten dem Team dadurch eröffnet werden.

\section*{Ausblick}

Folgende Erweiterungen dieser Arbeit wären möglich:

\begin{description}
    \item[Testzeitraum] \hfill \\ Die Entwicklung des Teams und des Dashboards mit den Metriken könnte noch über einen längeren Zeitraum verfolgt werden, um ausführlichere Rückschlüsse über die Effektivität und den Nutzen zu ziehen.
    \item[Testumfang] \hfill \\ Der Testumfang kann auf mehrere Teams erweitert werden, um ein noch umfangreicheres Feedback zu bekommen.
    \item[Systeme] \hfill \\ Mehr unterstützte Systeme kann die Akzeptanz der Software erhöhen.
    \item[Metriken] \hfill \\ Auch neue Metriken erhöhen die Akzeptanz der Software und erlauben es Teams, noch mehr Einsicht in ihre Prozesse und Systeme zu erlangen.
    \item[Geschäftsebenen] \hfill \\ Das Dashboard könnte noch weiteren Ebenen im Unternehmen bereitgestellt werden, zum Beispiel dem Management, um eine Übersicht über den Gesamtprozess zu erlangen.
\end{description}
