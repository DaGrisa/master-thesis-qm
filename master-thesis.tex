\documentclass[a4paper,12pt,twoside]{scrreprt}
% Autor der Vorlage: Klaus Rheinberger, FH Vorarlberg
% 2017-02-20

%% Hilfe: z.B.
% empfohlener Einstieg: http://latex.tugraz.at/  
% https://de.wikibooks.org/wiki/LaTeX-Kompendium:_Schnellkurs:_Erste_Schritte
% https://de.wikibooks.org/wiki/LaTeX-Kompendium:_Schnellkurs
% https://de.wikibooks.org/wiki/LaTeX-Kompendium

%% Pakete:
% Der Befehl \usepackage[latin9]{inputenc} ermöglicht die direkte Angabe von Umlauten. Übrigens lässt sich so auch das Euro-Zeichen direkt eingeben. Auf Betriebssystemen, wie zum Beispiel allen neueren Linux-Distributionen, verwendet man statt \usepackage[latin9]{inputenc} besser \usepackage[utf8]{inputenc}, auf Applesystemen verwendet man \usepackage[macce]{inputenc} (oder das für ältere Modelle gültige \usepackage[applemac]{inputenc}).
\usepackage[utf8]{inputenc}
\usepackage[T1]{fontenc}    % Silbentrennung bei Sonderzeichen
\usepackage{graphicx}       % Bilder einbinden
\usepackage{lscape}         % Querformat
\usepackage{pdfpages}
\usepackage[ngerman]{babel} % Deutsche Sprachanpassungen
\usepackage{footnote}       % footer citations
\usepackage{csquotes}       % When using babel or polyglossia with biblatex, loading csquotes is recommended to ensure that quoted texts are typeset according to the rules of your main language.
\usepackage{acronym}  % für optionales Abkürzungsverzeichnis
\usepackage[linktocpage=true]{hyperref} % Links z. B. \href{https://www.wikibooks.org}{Wikibooks home}
\usepackage{caption} % Abbildungslegenden
\usepackage{booktabs} % schönere horizontale Trennlinien für Tabellen 
\captionsetup{format=hang, justification=raggedright}
\usepackage[style=verbose,backend=bibtex]{biblatex}   % Literaturverweise
% biblatex comes with a variety of built-in bibliography/citation style families (numeric, alphabetic, authoryear, authortitle, verbose), and there's a growing number of custom styles:
% https://de.sharelatex.com/learn/Biblatex_citation_styles
% https://de.sharelatex.com/learn/Biblatex_bibliography_styles
\addbibresource{bibliography.bib}    % Zotero-Beispiele.bib ist die verwendete Bibtex-Datei 
% Anstatt die Bibtex-Datei selber zu erstellen, kann sie z. B. aus einer Zotero-Sammlung zu BibTeX exportiert werden.

% float package
\usepackage{float}

% fußnotennummerierung nicht neu beginnen in jedem kapitel
\usepackage{chngcntr}
\counterwithout*{footnote}{chapter}

% enumitem for enumarations
\usepackage{enumitem}
\setlist[1]{itemsep=0pt,parsep=0px}

% use for side-by-side figures
\usepackage{subfigure}

%% Einstellungen:
\setcounter{secnumdepth}{4}
\setcounter{tocdepth}{4}   % Tiefe der Gliederung im In haltsverzeichnis


%% ERSETZEN VON ECKIGEN KLAMMERN:
% Ersetzen Sie den Text in den eckigen Klammern!

\begin{document}

% Titelblatt:
% \newpage\mbox{}\newpage

% force output to a right page
\cleardoublepage{}
\thispagestyle{empty}
\begin{titlepage}
  \begin{flushright}
  \includegraphics[width=0.4\linewidth]{Logo-A3}
  \end{flushright}
  \begin{flushleft}
  \section*{Agile Metriken}
  \subsection*{Qualitätssicherung in agilen Teams}
  \vspace{1.5cm}
  
  Masterarbeit\\
  zur Erlangung des akademischen Grades
  \vspace{0.5cm}
  
  \textbf{Master of Science (MSc)}

  \vspace{2cm}
  Fachhochschule Vorarlberg\newline
  Informatik

  \vspace{1cm}
  
  Betreut von\newline
  Prof.\ Dr.\ Michael Felderer
  
  \vspace{2cm}
  
  Vorgelegt von\newline
  Daniel Grießer\newline
  Dornbirn, Juli 2018
  \end{flushleft}
\end{titlepage}

\newpage
\chapter*{Präambel}

Der Verfasser der vorliegenden Arbeit bekennt sich zu einer geschlechtergerechten Sprachverwendung. Aufgrund der häufig vorkommenden Nennung einiger Personengruppen werden zugunsten der flüssigeren Lesbarkeit für diese sowohl männliche als auch weibliche ”Rollen“ vergeben; diese sind ausgewogen und stereotypen Rollenbildern möglichst entgegenwirkend vergeben worden:

\begin{itemize}
    \item Die Entwicklerin
    \item Der Kunde
    \item Die Managerin
    \item Der Teilnehmer
    \item Die Botschafterin
    \item Der Stakeholder
    \item Die Benutzerin
\end{itemize}

Folgende Fachbegriffe werden direkt aus dem Englischen geschlechtsneutral übernommen:

\begin{itemize}
    \item Scrum Master
    \item Product Owner
    \item Meta Scrum
\end{itemize}

% evtl. Widmung:
\newpage
\section*{[evtl. Widmung]}   % evtl. ersetzen durch \section*{Widmung}

[Text der Widmung]

% Kurzreferat:
\newpage
\section*{Kurzreferat}

\subsection*{[Deutscher Titel Ihrer Arbeit]}

[Text des Kurzreferats]


% Abstract:
\newpage
\section*{Abstract}
\subsection*{[English Title of your thesis]}

[text of the abstract]


% evtl. Vorwort:
\newpage
\section*{[evtl. Vorwort]}   % evtl. ersetzen durch \section*{Widmung}

[Text des Vorworts]


% Inhaltsverzeichnis:
\cleardoublepage\tableofcontents

\clearpage
\phantomsection\addcontentsline{toc}{chapter}{Abbildungsverzeichnis}
\listoffigures

\clearpage
\phantomsection\addcontentsline{toc}{chapter}{Tabellenverzeichnis}
\listoftables

% Abkürzungsverzeichnis:
\clearpage
\phantomsection\addcontentsline{toc}{chapter}{Abkürzungsverzeichnis}
\chapter*{Abkürzungsverzeichnis}
\begin{acronym}
 \acro{LOC}{Lines of Code}
 \acro{CLOC}{Changed Lines of Code}
 \acro{VCS}{Version Control System}
 \acro{PTS}{Project Tracking System}
 \acro{DoD}{Definition of Done}
 \acro{CI}{Continuous Integration}
 \acro{CD}{Continuous Delivery}
 \acro{APM}{Application Performance Monitoring}
 \acro{BI}{Business Intelligence}
 \acro{IEEE}{Institute of Electrical and Electronics Engineers}
 \acro{MTTF}{Mean Time to Failure}
 \acro{MTTR}{Mean Time to Release}
 \acro{GQM}{Goal Question Metric}
 \acro{NASA}{National Aeronautics and Space Administration}
 \acro{NoSQL}{Not only SQL}
 \acro{QS}{Qualitätssicherung}
 \acro{SoS}{Scrum of Scrums}
 \acro{LeSS}{Large Scale Scrum}
 \acro{PBR}{Product Backlog Refinement}
 \acro{EAT}{Executive Action Team}
 \acro{EMS}{Executive MetaScrum}
\end{acronym}


%% Die Kapitelstruktur ist mit der Betreuungsperson abzustimmen!
\chapter{Einleitung}

Dieses Kapitel gibt eine kurze Übersicht über den Aufbau und die Motivation hinter dieser Arbeit.

\section{Zielsetzung}

Agile Prozesse, insbesondere Scrum, basieren auf Empirismus (siehe Abschnitt~\ref{section:scrum}).
Das bedeutet, dass sich der Prozess durch Reflexion verbessert und auch Veränderung zulässt.
Um diese Reflexion zu vereinfachen, ist es hilfreich, gewisse Kennzahlen des Prozesses und des Produktes grafisch darzustellen.
Im Speziellen Metriken können eine gute qualitative Auskunft über den aktuellen Status geben.
Ziel dieser Arbeit soll es sein, Metriken zu ermitteln, die qualitative Schwachstellen im Entwicklungsprozess oder im Softwareprodukt abbilden können.
Weiters sollen diese Metriken gesammelt und grafisch dargestellt werden.
Sie können aus Daten generiert werden, die bei der tagtäglichen Arbeit in den jeweiligen Systemen erzeugt werden.
Solche Systeme reichen von \acfp{VCS}, über \acfp{PTS} und \acf{CI} / \acf{CD}, bis hin zu \acf{APM}.
Diese Daten können meist über Schnittstellen abgefragt und anschließend aggregiert abgelegt werden.
Umgesetzt wird das Ganze in einem relativ jungen, aber im Scrum Prozess bereits weit fortgeschrittenen Scrum-Team.

\clearpage
\section{Aufbau der Arbeit}

Einen Einblick in die unterschiedlichen Themen und die theoretischen Hintergründe dieser Arbeit gibt das Kapitel ``Stand der Technik''.
Zuerst werden die Grundsäulen der agilen Softwareentwicklung, das agile Manifest und die agilen Prinzipien, genauer erklärt.
Darauf folgend wird auf Scrum eingegangen und zusätzlich Ansätze für Scrum in mehreren Teams erklärt.
Anschließend wird zu Qualität übergegangen, im Speziellen Software- und Prozessqualität.
Basierend auf den beiden vorherigen Themen, wird dann genauer auf Metriken eingegangen, was auch der Hauptteil dieses Kapitels darstellt.
Im ersten Teil werden Metriken aus den unterschiedlichsten Systemen vorgestellt und wie eigene Metriken erstellt werden können.
Danach folgen Hinweise zur Veröffentlichung von Metriken und der Messung von Agilen Prinzipien.
Am Schluss folgen noch ein Überblick über Qualitätsmodelle und das \ac{GQM}-Modell, sowie das \ac{FCM}-Modell im Detail.
\\
Im Kapitel ``Vorgehensweise'' wird beschrieben, wie bei der Bestimmung der relevanten Metriken für das Team, bei der Erstellung der Software und bei der Evaluierung der Ergebnisse vorgegangen wird.
\\
Das Kapitel ``Umsetzung'' zeigt dann, wie der Titel schon sagt, die Umsetzung der Lösung.
Anfangs werden die Gegebenheiten erläutert, in der die Lösung eingesetzt wird.
Dann wird mit der Identifizierung der Metriken gestartet und diese anschließend in der entwickelten Software gesammelt.
Zuletzt wird noch genauer auf die Darstellung eingegangen.
\\
Am Ende folgt das Kapitel ``Evaluierung'', in dem die Ergebnisse qualitativ in Form von Interviews und quantitativ in Form der Metriken evaluiert werden.
In den Kapiteln ``Schlussfolgerungen'' und ``Zusammenfassung'' werden die Ergebnisse nochmal reflektiert, zusammengefasst und ein Ausblick für mögliche weitere Arbeiten gegeben.

\chapter{Stand der Technik}

\section{Agile Softwareentwicklung}

Diese Arbeit dreht sich um agile Teams, deshalb ist es essentiell, zu verstehen, was der Gedanke hinter dem agilen Entwicklungsansatz ist.
Seinen Ursprung hat das Ganze, als sich 2001 ein paar schlaue Köpfe zusammengeschlossen haben und das sogenannte agile Manifest, sowie die agilen Prinzipien aufgestellt haben.
Ziel war es, eine Alternative zu den bisherigen, schwergewichtigen und von Dokumentation getriebenen Softwareentwicklungs-Methodologien zu finden.

\subsection{Agiles Manifest}

Das agile Manifest ist er Grundbaustein aller agilen Vorgehensmodelle:

\begin{quote}Wir erschließen bessere Wege, Software zu entwickeln,
indem wir es selbst tun und anderen dabei helfen.
Durch diese Tätigkeit haben wir diese Werte zu schätzen gelernt: \newline
\begin{center}
Individuen und Interaktionen mehr als Prozesse und Werkzeuge \newline
Funktionierende Software mehr als umfassende Dokumentation \newline
Zusammenarbeit mit dem Kunden mehr als Vertragsverhandlungen \newline
Reagieren auf Veränderung mehr als das Befolgen eines Plans \newline
\end{center}
Das heißt, obwohl wir die Werte auf der rechten Seite wichtig finden,
schätzen wir die Werte auf der linken Seite höher ein.\end{quote}\cite{agile_manifest}

\subsection{Agile Prinzipien}

Die agile Softwareentwicklung folgt diesen zwölf Prinzipien:

\begin{quote}Unsere höchste Priorität ist es, den Kunden durch frühe und kontinuierliche Auslieferung wertvoller Software zufrieden zu stellen.

Heisse Anforderungsänderungen selbst spät in der Entwicklung willkommen. Agile Prozesse nutzen Veränderungen zum Wettbewerbsvorteil des Kunden.

Liefere funktionierende Software regelmäßig innerhalb weniger Wochen oder Monate und bevorzuge dabei die kürzere Zeitspanne.

Fachexperten und Entwickler müssen während des Projektes täglich zusammenarbeiten.

Errichte Projekte rund um motivierte Individuen. Gib ihnen das Umfeld und die Unterstützung, die sie benötigen und vertraue darauf, dass sie die Aufgabe erledigen.

Die effizienteste und effektivste Methode, Informationen an und innerhalb eines Entwicklungsteams zu übermitteln, ist im Gespräch von Angesicht zu Angesicht.

Funktionierende Software ist das wichtigste Fortschrittsmaß.

Agile Prozesse fördern nachhaltige Entwicklung. Die Auftraggeber, Entwickler und Benutzer sollten ein gleichmäßiges Tempo auf unbegrenzte Zeit halten können.

Ständiges Augenmerk auf technische Exzellenz und gutes Design fördert Agilität.

Einfachheit -\phantom{}- die Kunst, die Menge nicht getaner Arbeit zu maximieren -\phantom{}- ist essenziell.

Die besten Architekturen, Anforderungen und Entwürfe entstehen durch selbstorganisierte Teams.

In regelmäßigen Abständen reflektiert das Team, wie es effektiver werden kann und passt sein Verhalten entsprechend an.\end{quote}\cite{agile_principles}

\newpage
\section{Scrum (Quellenangabe!)}

Das Scrum Framework ist eine solche agile Softwareentwicklungs-Methodologie. 
Scrum basiert auf Empirismus, also der Theorie, dass Wissen aus Erfahrung erlangt wird und Entscheidungen auf Basis dieses Wissens getroffen werden. 
Die drei Grundsäulen einer solchen empirischen Prozesskontrolle sind:

\begin{description}
  \item[Transparenz] \hfill \\ Signifikante Aspekte des Prozesses müssen für alle sichtbar sein.
  \item[Inspektion] \hfill \\ Artefakte müssen regelmäßig inspiziert werden, aber dieser Vorgang darf der Arbeit selbst nicht im Weg stehen.
  \item[Adaption] \hfill \\ Weicht ein oder mehrere Aspekte eines Prozesses von seinen akzeptablen Limits ab, muss dieser so früh wie möglich angepasst werden.
\end{description}

Das Scrum Framework (Abbildung~\ref{fig:scrum_framework}) besteht aus drei Rollen, fünf Ereignissen und drei Artefakten.

\begin{itemize}
  \item \textbf{Rollen}
  \begin{itemize}
    \item \textbf{Development Team}: Selbstorganisiertes Team, das am Produkt arbeitet.
    \item \textbf{Scrum Mater}: Verantwortlich dafür, sicherzustellen, dass Scrum verstanden und gelebt wird.
    \item \textbf{Product Owner}: Verantwortlich den Wert des Produktes und die Arbeit des Development Teams zu maximieren.
  \end{itemize}
  \item \textbf{Ereignisse}
  \begin{itemize}
    \item \textbf{Sprint}: Ist das Herz von Scrum: eine Timebox von 2 bis 4 Wochen, in dem ein fertiges, verwendbares und potentiell releasebares Produkt-Inkrement entwickelt wird.
    \item \textbf{Sprint Planning}: Planung eines Sprints. Hier commited sich das Scrum Team, eine gewisse Anzahl an Aufgaben im kommenden Sprint abzuarbeiten.
    \item \textbf{Daily Scrum}: Tägliches, zeitlich begrenztes Meeting, bei dem von jedem Teammitglied folgende drei Fragen beantwortet werden:
    \begin{enumerate}
      \item Was habe ich gemacht?
      \item Was werde ich machen?
      \item Was behindert mich bei meiner Arbeit?
    \end{enumerate}
    \item \textbf{Sprint Review}: Abschluss eines Sprints. Hier präsentiert das Team dem Product Owner die Ergebnisse des letzten Sprints.
    \item \textbf{Sprint Retrospective}: Das Team reflektiert den Sprint-Ablauf und ergreift Maßnahmen, um den Prozess weiter zu verbessern.
  \end{itemize}
  \item \textbf{Artefakte}
  \begin{itemize}
    \item \textbf{Product Backlog}: Ist eine Sammlung von möglichen Aufgaben für das Team am Produkt. Sollte einen Ausblick auf die zukünftige Entwicklung des Produktes geben. Oben im Product Backlog befinden sich die bereits fein geplanten Aufgaben, weiter unten die groben.
    \item \textbf{Sprint Backlog}: Entspricht den Aufgaben, die vom Team in den Sprint genommen und dem Product Owner zugesagt wurden.
    \item \textbf{Increment}: Entsteht am Ende eines jeden Sprints und ist eine lauffähige Version des Produkts, die releasefähig ist.
  \end{itemize}
\end{itemize}

\begin{savenotes}
  \begin{figure}[H] 
    \centering
    \includegraphics[width=0.9\textwidth]{img/scrum-framework.png}
    \caption[Scrum Framework]{Scrum Framework~\footcite{scrum_framework}}\label{fig:scrum_framework}
  \end{figure}
\end{savenotes}

\subsection[Scrum in mehreren Teams]{Scrum in mehreren Teams~\footcite[vgl.][S.172ff]{scrum_kurz_gut_2013}}

Scrum beschreibt eine agile Vorgehensweise für ein Team (ein Team entwickelt ein Produkt).
In der Realität existieren aber oft mehrere Teams und/oder mehrere Produkte. 
Dahingehend muss die Organisation der unterschiedlichen Scrum Teams individuell angepasst werden.
Für die Trennung der Teams gibt es unterschiedliche Ansätze:
\begin{description}
  \item[Trennung nach Organisationseinheiten] \hfill \\ Die Teams werden entlang der Abteilungsstruktur einer Organisation getrennt. Aus Scrum-Sicht macht das nicht immer Sinn, da bei der Umsetzung eines Features Abhängigkeiten zu anderen Teams bestehen (keine cross-funktionalen Teams).
  \item[Trennung nach Komponenten (Komponenten-Teams)] \hfill \\ Die technischen Komponenten werden den Teams zugeteilt, was ebenfalls zu Abhängigkeiten zu anderen Teams führt und eine gute Abstimmung zwischen den Teams voraussetzt.
  \item[Trennung nach fachlichen Themen (Feature-Teams)] \hfill \\ Jedes Team entwickelt, unabhängig von den anderen Teams, eine fachliche Komponente. Diese Variante erfüllt die Forderung des Scrum Frameworks nach cross-funktionalen Teams, weshalb bei dieser Form die Abstimmung zwischen den Teams am geringsten ist.
\end{description}

\begin{savenotes}
  \begin{figure}[H]
    \centering
    \subfigure[Feature-Teams]{\includegraphics[width=0.40\textwidth]{img/feature-teams.png}} 
    \subfigure[Komponenten-Teams]{\includegraphics[width=0.48\textwidth]{img/component-teams.png}} 
  \caption{Scrum Teams}\label{fig:Scrum Teams}
  \end{figure}
\end{savenotes}

In allen Varianten existieren aber pro Team unterschiedliche Software-Module und (agile) Prozesse, die unabhängig voneinander die Team-Qualität als gesamtes bestimmen.

\subsubsection{Scrum of Scrums}
\subsubsection{Less}
\subsubsection{Scale}

\clearpage
\section[Software-Qualität]{Software-Qualität~\footcite[vgl.][Kapitel 1.2]{hoffmann_software_qualitat_2013}}

Eine mögliche Definition von Software-Qualität findet sich in der DIN-ISO-Norm 9126:

\begin{quote}
  ``Software-Qualität ist die Gesamtheit der Merkmale und Merkmalswerte eines Software-Produkts, die sich auf dessen Eignung beziehen, festgelegte Erfordernisse zu erfüllen.''
\end{quote}

Wie aus dieser Definition schon erkennbar ist, gibt es viele unterschiedliche Kriterien, um die Qualität von Software zu bewerten.
Einige wesentliche Merkmale, um die Qualität von Software bewerten zu können, lassen sich in kunden- und herstellerorientierte Merkmale unterteilen:

\begin{description}
  \item[Kundenorientierte Merkmale] \hfill \\ Nach außen hin sichtbare Merkmale, die sich auf den kurzfristigen Erfolg der Software auswirken, da sie die Kaufentscheidung möglicher Kunden beeinflussen.
  \begin{description}
    \item[Funktionalität (Functionality, Capability)] \hfill \\ Beschreibt die Umsetzung der funktionalen Anforderungen. Fehler sind hier häufig Implementierungsfehler (sogenannte Bugs), welche durch Qualitätssicherung bereits in der Entwicklung entdeckt oder vermieden werden können. 
    \item[Laufzeit (Performance)] \hfill \\ Beschreibt die Umsetzung der Laufzeitanforderungen. Besonderes Augenmerk muss in Echtzeitsystemen auf dieses Merkmal gelegt werden.
    \item[Zuverlässigkeit (Reliability)] \hfill \\ Eine hohe Zuverlässigkeit ist in kritischen Bereichen, wie z.B. Medizintechnik oder Luftfahrt, unabdingbar. Erreicht werden kann diese aber nur durch die Optimierung einer Reihe anderer Kriterien.
    \item[Benutzbarkeit (Usability)] \hfill \\ Betrifft alle Eigenschaften eines Systems, die mit der Benutzer-Interaktion in Berührung kommen.
  \end{description}
  \item[Herstellerorientierte Merkmale] \hfill \\ Sind die inneren Merkmale, die sich auf den langfristigen Erfolg der Software auswirken und somit als Investition in die Zukunft gesehen werden sollten.
  \begin{description}
    \item[Wartbarkeit (Maintainability)] \hfill \\ Die Fähigkeit auch nach der Inbetriebnahme noch Änderungen an der Software vorzunehmen. Wird oft vernachlässigt, ist aber essentiell für langlebige Software und ein großer Vorteil gegenüber der Konkurrenz.
    \item[Transparenz (Transparency)] \hfill \\ Beschreibt, wie die nach außen hin sichtbare Funktionalität intern umgesetzt wurde. Gerade bei alternder Software, kann es zu einer Unordnung kommen, welche auch Software-Entropie (Grad der Unordnung) genannt wird.
    \item[Übertragbarkeit] \hfill \\ Wird auch Portierbarkeit genannt und beschreibt die Eigenschaft einer Software, in andere Umgebungen übertragen werden zu können (z.B. 32-Bit zu 64-Bit oder Desktop zu Mobile).
    \item[Testbarkeit (Testability)] \hfill \\ Testen stellt eine große Herausforderung dar, da oft auf interne Zustände zugegriffen werden muss oder die Komplexität die möglichen Eingangskombinationen vervielfacht. Aber gerade durch Tests können Fehler frühzeitig entdeckt und behoben werden.
  \end{description}
\end{description}

Je nach Anwendungsgebiet und den Anforderungen der Software haben die Merkmale unterschiedliche Relevanz und einige können sich auch gegenseitig beeinflussen, wie aus der Korrelationsmatrix in Abbildung~\ref{fig:scrum_framework} ersichtlich.
Dabei sind die positiv korrelierenden Merkmale mit ``+'' und die negativ korrelierenden mit ``-'' gekennzeichnet.

\begin{savenotes}
  \begin{figure}[H] 
    \centering
       \includegraphics[width=0.6\textwidth]{img/korrelationsmatrix-kriterien.png}
    \caption[Korrelationsmatrix Qualitätskriterien]{Korrelationsmatrix Qualitätskriterien~\footcite[][S. 11, Abb. 1.3]{hoffmann_software_qualitat_2013}}\label{fig:Korrelationsmatrix Qualitätskriterien}
  \end{figure}
\end{savenotes}

\newpage
\section{Metriken}

Eine Softwaremetrik wird vom \ac{IEEE} Standard 1061 von 1998 folgendermaßen definiert:
\begin{quote}
  ``Eine Softwarequalitätsmetrik ist eine Funktion, die eine Software-Einheit in einen Zahlenwert abbildet, welcher als Erfüllungsgrad einer Qualitätseigenschaft der Software-Einheit interpretierbar ist.''\footcite[vgl.][S.3]{ieee-1061}
\end{quote}

Vereinfacht gesagt, ist eine Metrik eine oder mehrere Kennzahlen, die mithilfe einer Funktion ein Qualitätsmerkmal in einen Zahlenwert abbilden.
Eine Kennzahl kann daher auch schon direkt eine Metrik sein, wenn sie in der Lage ist, ein gewünschtes Qualitätsmerkmal abzubilden.

\begin{savenotes}
  \begin{figure}[H] 
    \centering
       \includegraphics[width=0.6\textwidth]{img/software-development-lifecycle.png}
    \caption[Systeme im Softwareentwicklungsprozess]{Systeme im Softwareentwicklungsprozess~\label{fig:sdlc}}
  \end{figure}
\end{savenotes}

Im Entwicklungsprozess werden in den unterschiedlichen Systemen und Prozessschritten Daten erzeugt, die als Kennzahlen oder direkt als Metriken genutzt werden können.
Abbildung~\ref{fig:sdlc} zeigt die einzelnen Schritte und Systeme im Entwicklungsprozess.

\newpage
\subsection[Versionsverwaltung]{Versionsverwaltung~\footcite[vgl.][S.62ff]{davis_agile_2015}}

Das \ac{VCS} befindet sich nah an der Arbeit der Entwickler, da hier der Quellcode des Produkts verwaltet wird.
Daher können hier Daten darüber gesammelt werden, wie viel gearbeitet und auch wie viel zusammengearbeitet wird.
Um bestmögliche Daten zu bekommen, sollten verteilte Versionskontrollsysteme wie Git verwendet und mit Pull Requests gearbeitet werden.

\begin{description}
  \item[\ac{CLOC}] \hfill \\ Anzahl der geänderten Code Zeilen.
  \item[\ac{CLOC} pro Entwickler] \hfill \\ Anzahl der geänderten Zeilen im Quellcode pro Entwickler.
  \item[Commits] \hfill \\ Gesamtzahl an Commits in einem bestimmten Zeitraum.
  \begin{description}
    \item[Commits pro Entwickler] \hfill \\ Gesamtzahl an Commits in einem bestimmten Zeitraum pro Entwickler.
    \item[Kommentare pro Commit] \hfill \\ Anzahl der Kommentare pro Commit.
    \item[\ac{CLOC} pro Commit] \hfill \\ Anzahl der geänderten Zeilen im Quellcode pro Commit.
  \end{description}
  \item[Pull Requests] \hfill \\ Gesamtzahl an Pull Requests in einembestimmten Zeitraum.
  \begin{description}
    \item[Gemergte Pull Requests] \hfill \\ Anzahl erfolgreicher Pull Requests ineinem bestimmten Zeitraum.
    \item[Abgelehnte Pull Requests] \hfill \\ Anzahl abgelehnter Pull Requests in einem bestimmten Zeitraum.
    \item[Kommentare pro Pull Request] \hfill \\ Anzahl der Kommentare pro Pull Request.
  \end{description}
\end{description}

\newpage
\subsection[Projektmanagement]{Projektmanagement~\footcite[vgl.][S.37ff]{davis_agile_2015}}

In einem \ac{PTS} werden Aufgaben definiert und zugewiesen, Bugs verwaltet und Arbeitszeit mit Aufgaben verknüpft.
Hier können Daten über das Projektverständnis des Teams, die Geschwindigkeit und vor allem die Konsistenz der Arbeit gesammelt werden.
Um bestmögliche Daten erhalten zu können, gibt es folgende Empfehlungen:
\begin{itemize}[noitemsep]
  \item \ac{PTS} wird von allen genutzt
  \item Aufgaben mit möglichst vielen Tags versehen
  \begin{itemize}
    \item Aufgaben kategorisieren (nach ``gut'', ``ok'' und ``schlecht'')
  \end{itemize}
  \item Aufgaben schätzen
  \item gemeinsam eine \ac{DoD} festlegen
\end{itemize}
Jede Arbeit, die am \ac{PTS} vorbei geht, fällt später bei der Auswertung der Daten durch das Raster.
Durch das Taggen der Aufgaben können später Korrelationen ausgewertet werden, vor allem auch durch das Taggen, wie gut die Aufgabe abgelaufen ist.
Nur wenn die Aufgabe geschätzt ist, kann festgestellt werden, ob richtig geschätzt wurde oder wie viele Ausreißer es gibt. Dazu muss auch die Arbeitszeit auf der Aufgabe gespeichert werden.
Die \ac{DoD} hilft allgemein den Prozess zu verbessern und Rückläufe im Arbeitsablauf zu minimieren.

Dadurch ergeben sich folgende Kennzahlen aus einem \ac{PTS}:
\begin{description}
  \item[Burn Down] \hfill \\ Die Anzahl erledigte Arbeit über die Zeit. Liefert einen Richtwert, wo man sich gerade im Sprint befindet, verglichen zum Commitment.
  \item[Velocity] \hfill \\ Eine relative Messung der Konsistenz erledigter Arbeit über die Sprints.
  \item[Cummulative Flow] \hfill \\ Zeigt wie viel Aufgaben nach Status dem Team zugewiesen sind über die Zeit.
  \item[Lead Time] \hfill \\ Zeit zwischen Start und Abschluss einer Aufgabe, vor allem interessant bei Kanban.
  \item[Bug Counts] \hfill \\ Die Anzahl an Bugs über die Zeit.
  \begin{description}
    \item[Bug-Erzeugungsrate] \hfill \\ Anzahl Bugs nach Erstellungsdatum.
    \item[Bug-Fertigstellungsrate] \hfill \\ Anzahl Bugs nach Erledigungsdatum.
  \end{description}
  \item[Aufgaben-Volumen] \hfill \\ Die Anzahl der Aufgaben. Kann der Schätzung gegenübergestellt werden, um die Größe der Aufgaben oder ungeplante Arbeit aufzuzeigen.
  \item[Aufgaben-Rückfälligkeit] \hfill \\ Zeigt auf, wie oft Aufgaben im Arbeitsablauf rückwärts gehen.
\end{description}

\subsection[Kontinuierliche Integration und Auslieferung]{Kontinuierliche Integration und Auslieferung~\footcite[vgl.][S.84ff]{davis_agile_2015}}

\ac{CI}- und \ac{CD}-Systeme stellen sicher, dass die erstellte Software zu jedem Zeitpunkt auslieferbar ist, in dem sie zu definierten Zeitpunkten automatisch neu gebaut und ausgeliefert wird.
In einer solchen Build-Pipeline können sehr viel nützliche Daten erzeugt werden, vor allem mit Tools für statische Analysen (wie zum Beispiel SonarQube~\footcite[][]{sonarqube}).
Diese Systeme sind aber auch jene Elemente im Softwareentwicklungsprozess, die von Team zu Team am meisten variieren können.
Daher hängen die erzeugten Daten auch stark vom jeweiligen Setup ab.
Grundsätzlich können aber folgende Kennzahlen aus diesen Systemen ermittelt werden:

\begin{description}
  \item[Build-Dauer] \hfill \\ Geschätzte und tatsächliche Dauer der Builds.
  \item[Build-Status] \hfill \\ Es können die Anzahl der erfolgreichen und fehlerhaften Builds gegenüber gestellt werden.
  \item[Build-Frequenz] \hfill \\ Wie oft wird ein Build ausgelöst.
  \item[Test Reports] \hfill \\ Anzahl erfolgreicher und fehlerhafter Tests, Gesamtdauer der Tests.
  \item[Code Coverage] \hfill \\ Wie viel Prozent des Quellcodes ist mit Tests abgedeckt.
  \item[Stresstests oder Benchmarking] \hfill \\ Wird oft im Build Prozess mit getestet mit Tools wie JMeter~\footcite[][]{jmeter} oder Gatling~\footcite[][]{gatling}.
\end{description}

\newpage
\subsection[Produktionssystem]{Produktionssystem~\footcite[vgl.][S.107ff]{davis_agile_2015}}

Daten aus den Produktionssystemen können gesammelte \ac{APM}- oder auch \ac{BI}-Kennzahlen sein.
Diese Kennzahlen ermöglichen Aussagen, ob die Kunden zufrieden sind und wie das System arbeitet.
Die \ac{BI}-Kennzahlen sollten möglichst nahe am Entwicklungsteam gehalten werden, damit es verstehen kann, wie die Kunden die Applikation nutzen.
Dazu können Frameworks wie StatsD~\footcite[][]{statsd} und Atlas~\footcite[][]{atlas} verwendet werden.
Im Produktionssystem können folgende Kennzahlen ermittelt werden:

\begin{description}
  \item[CPU Nutzung] \hfill \\ Auslastung der Prozessoren über die Zeit.
  \item[Heap Size] \hfill \\ Auslastung des Heap über die Zeit.
  \item[Fehlerraten] \hfill \\ Anzahl Fehler über die Zeit (kann aus dem Logging kommen).
  \item[Antwortzeiten] \hfill \\ Dauer der Verarbeitung bestimmter Anfragen.
  \item[Benutzeranzahl] \hfill \\ Anzahl gleichzeitiger Benutzer in der Applikation über die Zeit.
  \item[Aufenthaltsdauer] \hfill \\ Verweildauer der Benutzer auf bestimmten Seiten.
  \item[Conversion Rate] \hfill \\ Anzahl Benutzer die zu Kunden wurden.
  \item[Semantisches Logging] \hfill \\ Ermöglicht es, beim Logging strukturierte Daten auszugeben, zum Beispiel: was suchen Benutzer auf bestimmten Seiten.
  \item[Verfügbarkeit] \hfill \\ Verfügbarkeit der Applikation über die Zeit.
\end{description}

\subsection{Übersicht Kennzahlen im Entwicklungsprozess}

Die Metriken finden sich nochmal als Tabelle dargestellt und mit den dazugehötigen Fragen, die sie jeweils beantworten, im Anhang~\ref{appendix:metrics}.

\subsection[Veröffentlichung von Metriken]{Veröffentlichung von Metriken~\footcite[vgl.][S.177ff]{davis_agile_2015}}

Metriken können auf verschiedene Art und Weise veröffentlicht werden. Zwei mögliche Beispiele sind Dashboards oder Emails.
Grundsätzlich sollte beachtet werden, dass man sich bei der Veröffentlichung von Metriken innerhalb der Grenzen und Gewohnheiten des Unternehmens bewegen sollte.
Außerdem sollte auf folgende Punkte geachtet werden:

\begin{description}
  \item[Dashboards] \hfill
  \begin{itemize}[noitemsep]
    \item den Zugriff innerhalb der Firma nicht einschränken
    \begin{itemize}[noitemsep]
      \item aber als intern ansehen
    \end{itemize}
    \item muss nach den Bedürfnissen der Teams anpassbar sein
    \item Metriken werden als Werkzeug gesehen, nicht als Waffe (gegen andere Teams oder Personen)
    \item Page Tracking verwenden, um das Nutzungsverhalten zu verstehen
  \end{itemize}
  \item[Emails] \hfill
  \begin{itemize}[noitemsep]
    \item aus dem Dashboard optional machen (sonst landen sie schnell automatisch im Spam-Ordner)
    \item minimal erforderliche Daten, den Rest verlinken zum Dashboard
    \item den Richtigen Rhytmus finden (zwischen oft genug informieren und nerven)
  \end{itemize}
\end{description}

Arbeitet ein Unternehmen beispielsweise viel mit Reports via Email, dann kann ein reines Dashboard weniger Anerkennung finden. Hier könnte beispielsweise eine Übersicht per Mail versendet und mit Links zum Dashboard versehen werden.

\subsection{Allgemeine Metriken}

Es gibt einige allgemeine Metriken, die für jedes Scrum Team von Bedeutung sind.
Wie stark, kann jedes Team selbst entscheiden, aber sie sollten nicht aus den Augen verloren werden.

\begin{itemize}
  \item Metrik 1
\end{itemize}

\subsection[Eigene Metriken erstellen]{Eigene Metriken erstellen~\footcite[vgl.][S.127ff]{davis_agile_2015}}

Um eigene Metriken erstellen zu können sind 2 Dinge notwendig:
\begin{itemize}
  \item Daten
  \item eine Funktion, um die Metrik zu berechnen
\end{itemize}

Dabei sollte darauf geachtet werden,
\begin{itemize}
  \item dass man auf die Metrik reagieren kann (Dinge, die einen stören und die man nicht ändern kann, frustrieren oder demotivieren)
  \item dass sich die Metrik nach den Team-Grundsätzen und Kerngeschäften ausrichtet
  \item dass die Metrik für sich alleine stehen kann
\end{itemize}

\subsection[Agile Prinzipien messen]{Agile Prinzipien messen~\footcite[vgl.][S.201ff]{davis_agile_2015}}

Um die agilen Prinzipien messen zu können, muss zuerst herausgefunden werden, was die Kernaussagen dieser Prinzipien sind.
Dies kann zum Beispiel grafisch, durch die Erstellung einer Wortwolke, wie in Abbildung~\ref{fig:wordcloud_principles} ersichtlich, erreicht werden.

\begin{savenotes}
  \begin{figure}[H] 
    \centering
    \includegraphics[width=0.9\textwidth]{img/principles-wordcloud.png}
    \caption{Agile Prinzipien als Wortwolke}\label{fig:wordcloud_principles}
  \end{figure}
\end{savenotes}

Aus dieser Wortwolke heben sich neben den Begriffen ``development'' und ``software'' vor allem auch die Begriffe ``team'', ``processes'', ``effective'' und ``requirements'' hervor.
Mithilfe dieser Begriffe lassen sich folgende vier Punkte ableiten:

\begin{itemize}[noitemsep]
  \item Effektive Software
  \item Effektiver Prozess
  \item Effektives Team 
  \item Effektive Anforderungen 
\end{itemize}

Für jeden dieser vier Punkte sind Metriken aus den unterschiedlichsten Systemen anwendbar~\footcite[vgl.][S.219ff]{davis_agile_2015}:

\begin{description}
  \item[Effektive Software] \hfill
  \begin{itemize}[noitemsep]
    \item erfolgreiche / fehlerhafte Builds
    \item Business-Metriken
    \item Status der Applikation
    \begin{itemize}[noitemsep]
      \item Fehlerraten
      \item CPU/Speicher Auslastung
      \item Antwort- / Transaktionszeiten
      \item Heapgröße / Garbage Collection / Anzahl Threads
    \end{itemize}
  \end{itemize}
  \item[Effektiver Prozess] \hfill
  \begin{itemize}[noitemsep]
    \item Velocity
    \item \ac{PTS} und \ac{VCS} Kommentare
    \item erfolgeiche Releases
  \end{itemize}
  \item[Effektives Team] \hfill
  \begin{itemize}[noitemsep]
    \item Lead Time
    \item \ac{MTTR}
    \item Deploy-Frequenz
    \item fehlerhafte Builds
  \end{itemize}
  \item[Effektive Anforderungen] \hfill
  \begin{itemize}[noitemsep]
    \item Rückläufigkeit
    \item Lead Time
    \item \ac{MTTR}
    \item Velocity
  \end{itemize}
\end{description}

\subsection[\ac{GQM}]{\ac{GQM}~\footcite[][]{basili_goal_nodate}}

\ac{GQM} ist ein Modell, um geeignete Metriken für ein Softwareprojekt finden zu können und wurde ursprünglich von der \ac{NASA} entwickelt, um Fehler in bestimmen Projekten zu erkennen.
Der grundlegende Gedanke dahinter ist, dass Metriken ``top\mbox{-}down'' (von oben nach unten) definiert werden müssen.
Dieser Ansatz wird dadurch begründet, dass Metriken sehr viele Charakteristiken abbilden können, aber erst durch Modelle beziehungsweise Ziele richtig genutzt und interpretiert werden können.

Das Ergebnis dieses Modells hat drei Level:
\begin{description}
  \item[GOAL \mbox{-} konzeptuelles Level] \hfill \\ Es werden Ziele für besimmte Objekte definiert. Objekte können Produkte, Prozesse oder Ressourcen sein.
  \item[QUESTION \mbox{-} operatives Level] \hfill \\ Für jedes Ziel werden Fragen formuliert, die zur Beurteilung oder Erreichung beitragen. Sie versuchen den Grund für die Messung zu charakterisieren.
  \item[METRIC \mbox{-} quantitatives Level] \hfill \\ Jeder Frage werden Metriken zugeordnet, die dabei helfen sollen, sie quantitativ zu beantworten.
\end{description}

\begin{savenotes}
  \begin{figure}[H] 
    \centering
    \includegraphics[width=0.9\textwidth]{img/gqm.png}
    \caption{hierarchische Struktur des GQM Modells}\label{fig:gqm}
  \end{figure}
\end{savenotes}

Abbildung~\ref{fig:gqm} zeigt die hierarchische Struktur des GQM-Modells.

\subsubsection[Beispiel]{Beispiel~\footcite[][]{basili_goal_nodate}}

Anhand des Beispiels in Tabelle~\ref{tab:gqm-example} soll veranschaulicht werden, wie so ein GQM-Modell in der Praxis aussehen kann.
Angenommen wird, das Team will seine Zusammenarbeit verbessern. 
Dazu muss das Ziel folgende Punkte spezifizieren: Absicht, Prozess / Produkt / Ressource, Sichtweise und Problem.
Dieses Ziel kann anschließend durch Fragen verfeinert werden.
Aussagen über Zusammenarbeit geben zum Beispiel Pull Requests.
Diese Fragen wiederum können mit Metriken beantwortet werden.
Im Falle der Pull Requests zum Beispiel der Durchschnitt und die Standardabweichung der Kommentare pro Pull Request in jedem Sprint.

\begin{table}[H]
  \centering
  \begin{tabular}{llr}
  GOAL     & \begin{tabular}[c]{@{}l@{}}Absicht\\ Problem\\ Prozess\\ Sichtweise\end{tabular} & \begin{tabular}[c]{@{}l@{}}Verbessern\\ der Zusammenarbeit innerhalb des Teams\\ im Entwicklungsprozess\\ aus Sicht der Entwickler.\end{tabular} \\ \midrule
  QUESTION & \multicolumn{2}{l}{Wie arbeitet das Team mit Pull Requests?} \\
  METRIC   & \multicolumn{2}{l}{\begin{tabular}[c]{@{}l@{}}Anzahl Pull Requests\\ Anzahl Kommentare pro Pull Request *\\ Anzahl Entwickler pro Pull Request *\end{tabular}} \\ \midrule
  QUESTION & \multicolumn{2}{l}{Wie viele Entwickler arbeiten an den einzelnen Modulen?} \\
  METRIC   & \multicolumn{2}{l}{\begin{tabular}[c]{@{}l@{}}Anzahl Commits pro Entwickler pro Modul\\ Anzahl Entwickler pro Pull Request pro Modul *\end{tabular}} \\
  \multicolumn{3}{l}{\begin{tabular}[c]{@{}l@{}} \\ \small{* Durchschnitt und Standardabweichung pro Sprint}\end{tabular}} 
  \end{tabular}
  \caption{Beispiel GQM Modell}\label{tab:gqm-example}
\end{table}

\subsection{Umfrage (Quelle fehlt)}

Quellen zu wissenschaftlichen Umfragen finden, passende einfügen.
Grundsätzlich: Team wurden die gängigsten Metriken vorgestellt und die Fragen, die sie beantworten können.
Jene Metriken, die in einer Skala von 1-10 eine x und größer hatten, wurden in dieser Arbeit umgesetzt.
Am Ende wurden offene Fragen gestellt, falls ein Teilnehmer noch eine Metrik vermisste.
Zusätzlich wurden qualitative Fragen zur Umfrage gestellt, um ein Feedback zur Methode zu bekommen.

\chapter{Vorgehensweise}

Im Folgenden wird erklärt, wie bei der Erarbeitung der Arbeit vorgegangen wird.
Grob gesehen, werden zuerst Metriken ermittelt, diese dann in einer Software regelmäßig erzeugt, gespeichert und angezeigt.
Zuletzt wird das Ergebnis der Arbeit evaluiert.

\section{Metriken bestimmen}

Das Unternehmen, in dem die Tests durchgeführt werden, arbeitet seit rund einem Jahr nach dem Scrum Framework.
Als Basis zur Bestimmung der Metriken können daher die vorhandenen Retrospektiven genutzt werden, da diese eine Richtung vorgeben, in die sich das Team bewegen will.
Es können die meistgenutzten Schlagwörter aus den Fragestellungen der Retrospektive ermittelt werden und nach dem \ac{GQM}-Modell in Metriken abgebildet werden.
Das \ac{FCM}-Modell ist hier weniger geeignet, da die Teams in dem Unternehmen, in dem die Software getestet wird, nicht an einem Produkt arbeiten, sondern an Geschäftsprozessen und sie ebenfalls den Scrum-Prozess verbessern wollen, nicht nur ein Produkt.
\\
\\
Als Ergänzung zum \ac{GQM}-Modell wird zusätzlich das Team in einer Umfrage zu möglichen Metriken befragt.
Dabei werden die in dieser Arbeit erarbeiteten Metriken genauer vorgestellt und von den Teammitgliedern bewertet.
Dadurch soll zusätzlich die Möglichkeit gegeben werden, Metriken oder Probleme, die zuvor nicht genannt wurden, angeben zu können.
Außerdem können die Ergebnisse des \ac{GQM}-Modells mit denen der Umfrage verglichen werden.

\clearpage
\section{Software}

Im Rahmen dieser Arbeit wird eine Software entwickelt, die Qualitätsmetriken aus unterschiedlichen Systemen ermitteln und bereitstellen kann.
Dabei sollen die zuvor ermittelten Metriken automatisch erzeugt und gespeichert werden.

\subsection{Anforderungen}\label{vorgehen:software}

Die Anforderungen an die Software entstanden zum einen aus der gewünschten Funktionsweise und zum anderen aus den Gegebenheiten des Umfelds der Software (und dem Unternehmen, in dem sie getestet werden soll).

\begin{description}
    \item[Erweiterbarkeit] \hfill \\ Um einfach neue Systeme und Metriken bereitstellen zu können, muss bei der Architektur auf eine einfache Erweiterbarkeit geachtet werden.
    \item[Fehlertoleranz] \hfill \\ Ein Fehler in einem einzelnen System, das Daten bereitstellt, darf nicht zum Absturz der Software führen.
    \item[Umsetzung in Java] \hfill \\ Java ist in vielen Unternehmen verbreitet und stößt daher auf eine hohe Akzeptanz.
    \item[einzubindende Systeme] \hfill \\ Metriken können in BitBucket Server~\footcite{bitbucket_server}, JIRA~\footcite{jira}, Jenkins~\footcite{jenkins}, SonarQube~\footcite{sonarqube}, Icinga~\footcite{icinga} vorkommen, da sie vom Unternehmen eingesetzt werden, in dem die Tests stattfinden. Systeme in denen relevante Metriken ermittelt wurden, müssen unterstützt werden.
    \item[Speicherung und Darstellung] \hfill \\ Speicherung und Darstellung der Metriken erfolgt in einem Elastic Stack~\footcite{elastic_stack}, da dieser ebenfalls bereits im Unternehmen vorhanden ist, in dem sie Tests stattfinden.
\end{description}

\clearpage
\section{Evaluierung}

Nach der Fertigstellung der Software wird diese für Testzwecke in einem Unternehmen für mehrere Sprints eingesetzt, um über einen möglichst langen Zeitraum Metriken zu sammeln.
Parallel dazu wird den betroffenen Teammitgliedern der Umgang mit dem Dashboard näher gebracht.
\\
Die Evaluierung wird zum Einen quantitativ durchgeführt, da sich Metriken sehr gut dafür eignen.
Allerdings kann es sein, dass die Zeit, die für den Test zur Verfügung steht, einfach zu kurz ist, um eine Tendenz in den Metriken erkennen zu können.
\\
Daher wird zum Anderen noch eine qualitative Evaluierung durchgeführt, in Form eines Interviews mit ausgewählten Teammitgliedern.
Dies dient dazu, die Effektivität der eingesetzten Methoden ermitteln zu können und das subjektive Empfinden gegenüber der Nützlichkeit von Metriken im agilen Prozess zu erfragen.

\chapter{Umsetzung}

\section{Vorgehensmodell}

\subsection{Metriken Identifizieren}

\subsubsection{\ac{GQM}}

Um eine Vorauswahl an Metriken treffen zu können, wurden alle bisherigen Retrospektiven (es waren genau 15) analysiert und eine Topliste von Schlagwörtern der folgenden Fragestellungen aus den Retrospektiven erstellt:
\begin{enumerate}
    \item Welche guten Entscheidungen haben wir getroffen?
    \item Was haben wir gelernt?
    \item Was können wir besser machen?
    \item Was nervt uns noch immer?
\end{enumerate}

Dazu wurden die Ergebnisse in eine ElasticSearch Datenbank gespeichert und über eine sogenannte Terms Aggregation die wichtigsten Schlagwörter analysiert.
Bei der Indizierung werden die Wörter normalisiert, deshalb die teilweise andere Schreibweise (zum Beispiel wird aus Issue der Term issu).
Die Ergebnisse in Anhang~\ref{appendix:retros} sind für die einzelnen Punkte wie folgt interpretierbar:

\begin{description}
    \item[Welche guten Entscheidungen haben wir getroffen?] \hfill
    \begin{itemize}[noitemsep]
      \item Die ersten fünf Wörter lassen darauf schließen, dass das Team gut zusammenarbeitet, speziell bei der Wissensverteilung: Transparenz, Onboarding von neuen Themen und Pair-Programming.
      \item Ebenfalls lässt sich aus den weiteren Begriffen schließen, dass viel Wert auf Reviews, Daily und die Arbeitsweise an sich gelegt wird.
    \end{itemize}
    \item[Was haben wir gelernt?] \hfill
    \begin{itemize}[noitemsep]
      \item Auch hier spiegelt sich der Daten- und Kommunikationsfluss im Team wider.
      \item Die vielen Scrum Schlagwörter zeigen, dass die Retrospektiven richtig genutzt wurden, um den Prozess zu verbessern.
    \end{itemize}
    \item[Was können wir besser machen?] \hfill
    \begin{itemize}[noitemsep]
      \item Auch hier zeugen die Scrum Schlagwörter wieder von einer Prozessverbesserung und einem selbstreflektiven Verhalten.
      \item Die Dokumentation scheint teilweise noch ein Problem zu sein, diese kommt gleich zweimal vor.
      \item Issues scheinen teilweise nicht optimal zu sein. Da könnte das Schlagwort ``groß'' dazu passen.
      \item ``backlog'', ``blocked'' und ``àblauf'' lassen auf Probleme im Arbeitsablauf schließen.
    \end{itemize}
    \item[Was nervt uns noch immer?] \hfill
    \begin{itemize}[noitemsep]
      \item Hier lassen die Schlagwörter ``updat'', ``erreichbar'', ``infrastruktur'', ``jenkin'', ``test'' und ``umgebung'' auf ein Infrastruktur Problem schließen, welches das Team womöglich ausbremst.
      \item Die Schlagwörter ``lang'', ``groß'' und ``klar'' lassen auf Probleme mit Anforderungen beziehungsweise Stories schließen.
      \item ``apis'', ``dba'', ``laut'' und ``iso'' sind wahrscheinlich äußere Einflüsse, die bei der täglichen Arbeit stören.
    \end{itemize}
  \end{description}

Aus diesen Ergebnissen Lassen sich die \ac{GQM}-Modelle in Tabelle~\ref{tab:gqm-distraction},~\ref{tab:gqm-process} und~\ref{tab:gqm-issues} ableiten.

\begin{table}[H]
    \centering
    \begin{tabular}{llr}
    GOAL     & \begin{tabular}[c]{@{}l@{}}Absicht\\ Problem\\ Ressource\\ Sichtweise\end{tabular} & \begin{tabular}[c]{@{}l@{}}Verringerung\\ der Ablenkung\\ von Entwicklern\\ aus Sicht des Scrum Masters.\end{tabular} \\ \midrule
    QUESTION & \multicolumn{2}{l}{Wie viele Aufgaben erledigt das Team pro Sprint?} \\
    METRIC   & \multicolumn{2}{l}{Aufgaben-Volumen pro Sprint} \\ \midrule
    QUESTION & \multicolumn{2}{l}{Wie viele Aufgaben werden jeden Tag erledigt?} \\
    METRIC   & \multicolumn{2}{l}{\begin{tabular}[c]{@{}l@{}}erledigte Aufgaben pro Tag\\ Burn-down pro Tag\end{tabular}} \\ \midrule
    QUESTION & \multicolumn{2}{l}{Welche Tags haben als ``schlecht'' bewertete Aufgaben?} \\
    METRIC   & \multicolumn{2}{l}{\begin{tabular}[c]{@{}l@{}}Tags der Aufgaben, \\ um ``schlecht'' bewertete besser analysieren zu können\end{tabular}}
    \end{tabular}
    \caption{GQM\mbox{-}Modell \mbox{-} Ablenkung der Entwickler}\label{tab:gqm-distraction}
\end{table}

\begin{table}[H]
    \centering
    \begin{tabular}{llr}
    GOAL     & \begin{tabular}[c]{@{}l@{}}Absicht\\ Problem\\ Prozess\\ Sichtweise\end{tabular} & \begin{tabular}[c]{@{}l@{}}Optimierung\\ des Durchlaufes\\ im Entwicklungsprozess\\ aus Sicht des Scrum Teams.\end{tabular} \\ \midrule
    QUESTION & \multicolumn{2}{l}{Gibt es irgendwelche Engpässe im Prozess?} \\ 
    METRIC   & \multicolumn{2}{l}{Cumulative Flow} \\ \midrule
    QUESTION & \multicolumn{2}{l}{Wie lange dauert der Durchlauf einer Aufgabe?} \\
    METRIC   & \multicolumn{2}{l}{Lead Time}
    \end{tabular}
    \caption{GQM\mbox{-}Modell \mbox{-} Schwachstellen im Prozess}\label{tab:gqm-process}
\end{table}

\begin{table}[H]
    \centering
    \begin{tabular}{llr}
    GOAL     & \begin{tabular}[c]{@{}l@{}}Absicht\\ Problem\\ Ressource\\ Sichtweise\end{tabular} & \begin{tabular}[c]{@{}l@{}}Optimierung\\ der Aufgabengröße\\ im Backlog\\ aus Sicht des Product Owners.\end{tabular} \\ \midrule
    QUESTION & \multicolumn{2}{l}{Wie groß ist die Aufgabengröße im Durchschnitt?} \\
    METRIC   & \multicolumn{2}{l}{erledigte Story-Points / Aufgaben-Volumen} \\ \midrule
    QUESTION & \multicolumn{2}{l}{Wie lange dauert der Durchlauf einer Aufgabe?} \\
    METRIC   & \multicolumn{2}{l}{Lead-Time} \\ \midrule
    QUESTION & \multicolumn{2}{l}{Wie viele Aufgaben gehen im Entwicklungsprozess rückwärts?} \\
    METRIC   & \multicolumn{2}{l}{Aufgaben-Rückfälligkeit}
    \end{tabular}
    \caption{GQM\mbox{-}Modell \mbox{-} Aufgabengröße}\label{tab:gqm-issues}
\end{table}

\subsubsection{Umfrage im Team}

Zusätzlich zu der Analyse der Retrospektiven wurden dem Team Metriken und die Fragen, die damit beantwortet werden können, in Form einer Umfrage vorgestellt.
Die einzelnen Metriken wurden von den Teammitgliedern nach Wichtigkeit mit einer Skala von 1 bis 10 bewertet.
Anhang~\ref{appendix:questions} zeigt die Umfrage, wie sie den Teammitgliedern vorgelegt wurde und Anhang~\ref{appendix:answers} die dazugehörigen Antworten.
Die Ergebnisse wurden nach ihrem Druchschnittwert sortiert und die 10 als am wichtigsten bewerteten Metriken sind:

\begin{itemize}[noitemsep]
    \item \textbf{Burn Down (9,00)} \mbox{-} Erfüllt das Team seine Commitments? Plant das Team seine Arbeit realistisch?
    \item \textbf{Velocity (9,00)} \mbox{-} Wie konsistent arbeitet das Team?
    \item \textbf{Aufgaben-Volumen (8,57)} \mbox{-} Wie viel ungeplante Arbeit kam zum Sprint dazu? Wie groß ist die durchschnittliche Aufgabe? Gibt es Ausreißer?
    \item \textbf{Cumulative Flow (8,29)} \mbox{-} Gibt es Engpässe oder Schwachstellen im Prozess? Müssen gewisse Abläufe im Prozess optimiert werden?
    \item \textbf{Lead Time (8,14)} \mbox{-} Wie schnell können Aufgaben vom Team erledigt werden? Wie lange dauert die Umsetzung eines neuen Features?
    \item \textbf{Stresstests oder Benchmarking (7,86)} \mbox{-} Ist das Produkt auch noch unter Last verwendbar? Wie verändert sich die Leistung über die Zeit?
    \item \textbf{Code Coverage (7,71)} \mbox{-} Gibt es Module, die nicht oder schlecht getestet sind? Wie sieht die Entwicklung der Testabdeckung über die Zeit aus?
    \item \textbf{Bug Counts (7,57)} \mbox{-} Wie viele Fehler werden vom Team im Entwicklungsprozess übersehen? Wie viel ungeplante Arbeit kam zum Sprint dazu?
    \item \textbf{Aufgaben-Rückfälligkeit (7,57)} \mbox{-} Wie viele Aufgaben werden wieder in einen vorhergehenden Status gesetzt? Gibt es Probleme beim Verständnis der Aufgaben? Wie klar sind die Erwartungen des Teams an eine abgeschlossene Änderung (DoD)?
    \item \textbf{Bug-Erzeugungsrate (7,43)} \mbox{-} Wie viele Fehler wurden zu einem bestimmten Zeitpunkt erzeugt?
\end{itemize}

Zusätzlich wurde in den offenen Fragen am Ende zwiemal gefordert, dass die Flüchtigkeit von Anforderungen Sichtbar wird. 
Dies kann zum einen durch die oben genannte Aufgaben-Rückfälligkeit und andererseits durch folgende Metrik abgebildet werden:

\begin{itemize}[noitemsep]
    \item \textbf{Anforderungen-Flüchtigkeit} \mbox{-} Wie oft wurde die Anforderung der Aufgabe angepasst?
\end{itemize}

\section{Software}

Technologien, Plattform, etc.

\subsection{Architektur}

Abbildung~\ref{fig:position_architecture} zeigt die Position und Abbildung~\ref{fig:overview_architecture} die grobe Architektur der Software (Agile Metrics).
Die Software bildet eine Schnittstelle zwischen den einzelnen Systemen des Entwicklungsprozesses und dem System zur Darstellung der Metriken (in diesem Fall ElasticSearch und Kibana).

\begin{savenotes}
    \begin{figure}[H] 
        \centering
            \includegraphics[width=0.8\textwidth]{img/position-overview.png}
        \caption{Position der Software}\label{fig:position_architecture}
    \end{figure}
\end{savenotes}

\begin{savenotes}
    \begin{figure}[H] 
        \centering
            \includegraphics[width=0.8\textwidth]{img/architecture-overview.png}
        \caption{Übersicht der Software-Architektur}\label{fig:overview_architecture}
    \end{figure}
\end{savenotes}

\begin{description}
    \item[UI] \hfill \\ Bietet eine grafische Benutzeroberfläche zur Konfiguration.
    \item[Producer] \hfill \\ Sind Schnittstellen zu allen Systemen, die Messdaten erzeugen.
    \item[Crons] \hfill \\ Zeitsteuerung der Messdaten-Abfrage (z.B. täglich oder pro Sprint).
    \item[Metrics] \hfill \\ Hier können aus Messdaten direkt Metriken erstellt werden.
    \item[Consumer] \hfill \\ Sind Schnittstellen zu allen Systemen, die Messdaten und Metriken konsumieren.
\end{description}

\chapter{Evaluierung}

Um die Effektivität der erarbeiteten Lösung feststellen zu können, wird eine Evaluierung durchgeführt.
Die dargestellten Metriken bieten sich an, um eine quantitative Evaluierung durchzuführen.
Zusätzlich dazu werden ausgewählte Teammitglieder interviewt, um eine qualitative Evaluierung durchführen zu können.

\section{Quantitative Evaluierung}

Anhand der zeitlichen Entwicklung der gemessenen Metriken wäre eine quantitative Evaluierung möglich.
Allerdings ist einerseits der Zeitraum von 2{-}3 Sprints noch zu gering, um eine Entwicklung auf die Maßnahmen rückschließen zu können,
andererseits fehlen bei manchen Metriken die Werte vor der Inbetriebnahme, was eine Änderung ab der Inbetriebnahme ebenfalls schwer erkennen lässt.

\clearpage
\section{Qualitative Evaluierung}

Zur qualitativen Evaluierung werden Interviews mit dem Scrum-Master, dem Product Owner und einer Entwicklerin geführt.
Das ermöglicht einen Einblick aus Managementsicht (Product Owner) und aus Prozesssicht (Scrum Master und Entwicklerin), welche womöglich ganz unterschiedliche Anforderungen an das Produkt stellen.
Dabei soll herausgefunden werden, ob die richtigen Metriken ermittelt wurden und welche Metriken als besonders nützlich angesehen werden.
Ebenfalls wird nach einer nachweisbaren und spürbaren Qualitätsverbesserung gefragt.
Interessant ist auch noch wann und wie das Dashboard genutzt wird und wie es um die Benutzbarkeit steht.

\subsection{Interview-Fragen}

Folgende Interviewfragen sollen helfen, den roten Faden im Gespräch zu behalten.
Als Einstieg wird das Dashboard geöffnet, um es während dem Gespräch immer sichtbar vor sich zu haben.

\begin{enumerate}
    \item Wird das Dashboard von dir genutzt? Wenn ja, wann und wie nutzt du das Dashboard?
    \item Wie ist der Zugang und die Bedienbarkeit des Dashboards?
    \item Ist das Dashboard übersichtlich und klar eingeteilt?
    \item Rückblickend gesehen, wurden die richtigen Metriken ermittelt und auf dem Dashboard visualisiert?
    \item Welche Metriken auf dem Dashboard sind besonders nützlich oder werden oft genutzt?
    \item Ist bereits eine Qualitätsverbesserung im Prozess oder in einem Produkt spürbar? Oder sogar nachweisbar?
    \item Gibts es aus deiner Sicht Verbesserungspotential? Wo liegen aus deiner Sicht die Schwächen dieser Lösung?
\end{enumerate}

Die Interviews wurden mit Zustimmung der Interviewten aufgezeichnet, ein Transkript befindet sich in Anhang~\ref{appendix:transcript}.
Interne und vertrauliche Informationen wurden mit *** ersetzt.

\clearpage
\subsection{Interview-Antworten}

Auffallend bei den Antworten sind die unterschiedlichen Sichten auf das Dashboard.
Während der Product Owner das Dashboard mehr für Werbezwecke genutzt hat, um andere Bereiche auf die Möglichkeit von Metriken aufmerksam zu machen, hat es es der Scrum Master eher für die Langzeit-Sicht des Teams genutzt und seinen Fokus auf die Verbesserung des Prozesses gelegt.
Die Entwicklerin wiederum hatte den Fokus auf die kurzfristigen Metriken, wie den Bug Count, um schnell auf Probleme reagieren zu können.
Genutzt wurde es daher von allen Befragten, wichtiger ist aber, dass alle einen Nutzen darin sahen und es für sich bestmöglich nutzen können.
Bei der Bedienbarkeit sind einzelne Schwächen aufgetaucht, vor allem was die Detailansicht von Kennzahlen betrifft.
Die Einteilung wurde von allen als klar gesehen, eine Kennzeichnung der betroffenen Rollen (Scrum Master, Product Owner oder Entwicklerinnen) wäre eventuell noch sinnvoll.
Auch eine genauere Beschreibung der Ziele und gewisser Metriken, könnte Missverständnisse verhindern.
Die Auswahl der Metriken wurde als gut empfunden, manche Metriken machen erst Sinn, wenn sie mit anderen kombiniert werden.
Diese Kombination von Metriken war aber auch vor allem dem Scrum Master klar und wird in absehbarer Zukunft auch noch so umgesetzt.
Welche Metriken als besonders nützlich angesehen werden, hängt stark von der Rolle ab.
Wie bereits anfangs erwähnt, interessiert sich die Entwicklerin mehr für die kurzzeitigen Metriken, während das Interesse des Scrum Masters und Product Owners mehr bei den Langzeit-Metriken liegt.
Eine Qualitätsverbesserung ist noch nicht klar ersichtlich, da es ein noch zu kurzer Zeitraum ist, um darüber Aussagen treffen zu können.
Was aber von der Entwicklerin als spürbar erwähnt wurde, ist der Umgang mit den Bugs, die nun frühzeitig erkennt und abgearbeitet werden können.
Verbesserungspotential sehen alle noch bei der Nutzung der Metriken, was aber auch vor allem auf den zu kurzen Zeitraum der Tests zurückschließen ist.
Einige gute Vorschläge, wie das Beschreiben von Zielen, eine bessere Kategorisierung nach Rollen und das Interesse an neuen Metriken zeigt aber, dass das Dashboard durchaus genutzt und das Ziel dahinter, das Ganze als Wekrzeug für das gesamte Team zu nutzen, bereits erkannt wurde.

\chapter{Schlussfolgerungen}

In dieser Arbeit konnte in einer Fallstudie gezeigt werden, dass mithilfe der \ac{GQM}-Methodik, ergänzt durch eine Umfrage im entsprechenden Scrum-Team, Metriken ermittelt werden können, die es ermöglichen, die Schwachstellen in einem Produkt und im agilen Prozess in Zahlen zu fassen.
Diese Metriken ermöglichen es dem agilen Team, seine Fortschritte bei den Retrospektiven nachvollziehbar zu bewerten und weitere Maßnahmen zu treffen.
Außerdem werden durch die Kombination unterschiedlicher Metriken Korrelationen zwischen bestimmten Metriken nachgewiesen oder widerlegt.
Reichen die bekannten Metriken in der Literatur nicht aus, besteht die Möglichkeit, dass auch eigene Metriken erstellt werden.
\\
Die entwickelte Software hilft dabei, die Daten aus den unterschiedlichen Systemen im Entwicklungsprozess als Metriken aufzubereiten und zu speichern.
Dabei wurde bei der Architektur auf Fehlertoleranz und einfache Erweiterbarkeit geachtet, sodass ein einfacher Betrieb und eine Erweiterung der unterstützten Systeme und Metriken möglich ist.
Für einen einfachen Zugang und eine uneingeschränkte Erweiterbarkeit wurde der Quellcode der Software unter der quelloffenen MIT-Lizenz veröffentlicht.
Bei der Visualisierung von Metriken bietet Kibana eine geeignete Plattform, um aus den gespeicherten Metriken einfach Dashboards mit unterschiedlichen Visualisierungen bereitzustellen.
Dabei muss berücksichtigt werden, dass Entwicklerinnen, Scrum-Master und Product-Owner jeweils unterschiedliche Interessen an den Metriken haben.
Durch eine geeignete Gruppierung der Metriken ist es möglich, den unterschiedlichen Teammitgliedern auf einen Blick die wichtigsten Metriken anzuzeigen.
Dadurch wird das Dashboard auch regelmäßig genutzt und somit die Akzeptanz noch weiter erhöht.
Der Einsatz der Software ermöglicht den Aufbau einer zentralen Stelle, an der die Metriken aus allen relevanten System gesammelt, dargestellt und kombiniert werden.
\\
Schwachstellen wurden bei der Art der Darstellung mancher Metriken identifiziert.
So wurde zum Beispiel der Wunsch nach einer besseren Beschreibung der Metriken, um Diskussionen über deren Bedeutung zu vermeiden, geäußert.
Zusätzlich soll künftig eine detailliertere Beschreibung von Zielen helfen, die Absicht hinter Metriken zu erklären.
Bei manchen Darstellungen wurden noch zusätzliche Detailinformationen zu den einzelnen Datensätzen gewünscht, um Ausreißer leichter zu identifizieren.
Trotzdem wurde das erstellte Dashboard trotz des relativ kurzen Testzeitraums von nicht ganz drei Sprints gut angenommen.
Bei der Evaluierung wurde auch klar, dass die Teammitglieder bereits erkannten, wie sie das Dashboard persönlich am besten nutzen.
\\
Durch den Einsatz der entwickelten Software und der vorgestellten Modelle zur Identifizierung von relevanten Metriken, kann die Qualität in einem agilen Team dadurch erhöht werden, dass Qualitätsprobleme durch Metriken sichtbar gemacht und in den Retrospektiven Gegenmaßnahmen dafür getroffen werden können.

\chapter{Zusammenfassung}



\section{Ausblick}

Folgende Erweiterungen dieser Arbeit wären möglich:

\begin{description}
    \item[Testzeitraum] \hfill \\ Die Entwicklung des Teams und des Dashbords mit den Metriken könnte noch über einen längeren Zeitraum verfolgt werden, um ausführlichere Rückschlüsse über die Effektivität und den Nutzen zu ziehen.
    \item[Testumfang] \hfill \\ Der Testumfang kann auf mehrere Teams erweitert werden, um ein noch umfangreicheres Feedback zu bekommen.
    \item[Systeme] \hfill \\ Mehr unterstützte Systeme kann die Akzeptanz der Software erhöhen.
    \item[Metriken] \hfill \\ Auch neue Metriken erhöhen die Akzeptanz der Software und können es Teams erlauben, noch mehr Einsicht in ihre Prozesse und Systeme zu erlangen.
    \item[Geschäftsebenen] \hfill \\ Das Dashbord könnte noch weiteren Ebenen im Unternehmen bereitgestellt werden, zum Beispiel dem Management, um eine Übersicht über den Gesamtprozess zu erlangen.
\end{description}


% Literaturverzeichnis:
\clearpage
\phantomsection{}
\addcontentsline{toc}{chapter}{Literaturverzeichnis}
\printbibliography{}

\appendix
\addchap{Anhang}
\refstepcounter{chapter}

\section{Metriken aus dem Entwicklungsprozess}\label{appendix:metrics}

\begin{table}[H]
    \centering
    \begin{tabular}{p{6.5cm}p{8cm}} \toprule
    \textbf{\ac{CLOC}} & Anzahl der geänderten Zeilen im Quellcode. \\
    \multicolumn{2}{p{14.5cm}}{\textit{Wie viele Änderungen passieren in der Codebasis? \newline Wo finden die meisten Änderungen statt?}} \\ \midrule
    \textbf{\ac{CLOC} pro Entwickler} & Anzahl der geänderten Zeilen im Quellcode pro Entwickler. \\ 
    \multicolumn{2}{p{14.5cm}}{\textit{Wie viel Code ändert jeder im Team? \newline  Wer ist wie oft in welchem Modul?}} \\ \midrule
    \textbf{\ac{CLOC} pro Commit} & Anzahl der geänderten Zeilen im Quellcode pro Commit. \\ 
    \multicolumn{2}{p{14.5cm}}{\textit{Wie groß sind die Commits?}} \\ \midrule
    \textbf{Commits} & Gesamtzahl an Commits in einem bestimmten Zeitraum. \\ 
    \multicolumn{2}{p{14.5cm}}{\textit{Wie viel Änderungen wurden im Quellcode vorgenommen?}} \\ \midrule
    \textbf{Commits pro Entwickler} & Gesamtzahl an Commits in einem bestimmten Zeitraum pro Entwickler. \\ 
    \multicolumn{2}{p{14.5cm}}{\textit{Wie viel Änderungen wurden im Quellcode von einem Entwickler vorgenommen?}} \\ \midrule
    \textbf{Kommentare pro Commit} & Anzahl der Kommentare pro Commit. \\ 
    \multicolumn{2}{p{14.5cm}}{\textit{Wer arbeitet zusammen? \newline Wie viel wird zusammengearbeitet?}} \\ \midrule
    \textbf{Pull Requests} & Gesamtzahl an Pull Requests in einem bestimmten Zeitraum. \\ 
    \multicolumn{2}{p{14.5cm}}{\textit{Wird mit Pull Requests gearbeitet? \newline Werden Reviews gemacht?}} \\ \midrule
    \textbf{Gemergte Pull Requests} & Anzahl erfolgreicher Pull Requests in einem bestimmten Zeitraum. \\ 
    \multicolumn{2}{p{14.5cm}}{\textit{Wie oft werden erfolgreiche Änderungen in die Codebasis übernommen?}} \\ \midrule
    \textbf{Abgelehnte Pull Requests} & Anzahl abgelehnter Pull Requests in einem bestimmten Zeitraum. \\ 
    \multicolumn{2}{p{14.5cm}}{\textit{Wie oft werden Änderungen an der Codebasis abgelehnt? \newline Wie klar sind die Erwartungen des Teams an eine abgeschlossene Änderung (\ac{DoD})?}} \\ \midrule
    \textbf{Kommentare pro Pull Request} & Anzahl der Kommentare pro Pull Request. \\ 
    \multicolumn{2}{p{14.5cm}}{\textit{Wer arbeitet zusammen? \newline Wie viel wird zusammengearbeitet?}} \\ \bottomrule
    \end{tabular}
    \caption{Kennzahlen aus dem \ac{VCS}}\label{metrics-table-vcs}
  \end{table}
  
  \begin{table}[H]
    \centering
    \begin{tabular}{p{5cm}p{9.5cm}} \toprule
    \textbf{Burn Down} & Die Anzahl erledigte Arbeit über die Zeit. Liefert einen Richtwert, wo man sich gerade im Sprint befindet, verglichen zum Commitment. \\
    \multicolumn{2}{p{14.5cm}}{\textit{Erfüllt das Team seine Commitments? \newline Plant das Team seine Arbeit realistisch?}} \\ \midrule
    \textbf{Velocity} & Eine relative Messung der Konsistenz erledigter Arbeit über die Sprints. \\
    \multicolumn{2}{p{14.5cm}}{\textit{Wie konsistent arbeitet das Team?}} \\ \midrule
    \textbf{Cumulative Flow} & Zeigt wie viel Aufgaben nach Status dem Team zugewiesen sind über die Zeit. \\
    \multicolumn{2}{p{14.5cm}}{\textit{Gibt es Engpässe oder Schwachstellen im Prozess? \newline Müssen gewisse Abläufe im Prozess optimiert werden?}} \\ \midrule
    \textbf{Lead Time} & Zeit zwischen Start und Abschluss einer Aufgabe, vor allem interessant bei Kanban. \\
    \multicolumn{2}{p{14.5cm}}{\textit{Wie schnell können Aufgaben vom Team erledigt werden? \newline Wie lange dauert die Umsetzung eines neuen Features?}} \\ \midrule
    \textbf{Bug Counts} & Die Anzahl an Bugs über die Zeit. \\
    \multicolumn{2}{p{14.5cm}}{\textit{Wie viele Fehler werden vom Team im Entwicklungsprozess übersehen? \newline Wie viel ungeplante Arbeit kam zum Sprint dazu?}} \\ \midrule
    \textbf{Bug-Erzeugungsrate} & Anzahl Bugs nach Erstellungsdatum. \\
    \multicolumn{2}{p{14.5cm}}{\textit{Wie viele Fehler wurden zu einem bestimmten Zeitpunkt erzeugt?}} \\ \midrule
    \textbf{Bug-Fertigstellungsrate} & Anzahl Bugs nach Erledigungsdatum. \\
    \multicolumn{2}{p{14.5cm}}{\textit{Wie viele Fehler wurden zu einem bestimmten Zeitpunkt beseitigt?}} \\ \midrule
    \textbf{Aufgaben-Volumen} & Ist die Anzahl der Aufgaben und kann der Schätzung gegenübergestellt werden, um die Größe der Aufgaben oder ungeplante Arbeit aufzuzeigen. \\
    \multicolumn{2}{p{14.5cm}}{\textit{Wie viel ungeplante Arbeit kam zum Sprint dazu? \newline Wie groß ist die durchschnittliche Aufgabe? Gibt es Ausreißer?}} \\ \midrule
    \textbf{Aufgaben-Rückfälligkeit} & Zeigt auf, wie oft Aufgaben im Arbeitsablauf rückwärts gehen. \\
    \multicolumn{2}{p{14.5cm}}{\textit{Wie viele Aufgaben werden wieder in einen vorhergehenden Status gesetzt? \newline Gibt es Probleme beim Verständnis der Aufgaben? \newline Wie klar sind die Erwartungen des Teams an eine abgeschlossene Änderung (DoD)?}} \\ \bottomrule
    \end{tabular}
    \caption{Kennzahlen aus dem \ac{PTS}}\label{metrics-table-pts}
  \end{table}
  
  \begin{table}[H]
    \centering
    \begin{tabular}{p{6.5cm}p{8cm}} \toprule
    \textbf{Build-Dauer} & Geschätzte und tatsächliche Dauer der Builds. \\
    \multicolumn{2}{p{14.5cm}}{\textit{Wie lange dauert es ein Software-Artefakt zu erstellen? \newline Wie verändert sich die Dauer der Erstellung eines Software-Artefakts über die Zeit?}} \\ \midrule
    \textbf{Build-Status} & Es können die Anzahl der erfolgreichen und fehlerhaften Builds gegenüber gestellt werden. \\
    \multicolumn{2}{p{14.5cm}}{\textit{Gibt es ein Problem im Freigabeprozess?}} \\ \midrule
    \textbf{Build-Frequenz} & Wie oft wird ein Build ausgelöst. \\
    \multicolumn{2}{p{14.5cm}}{\textit{Wird oft genug ein neues Software-Artefakt erstellt?}} \\ \midrule
    \textbf{Test Reports} & Anzahl erfolgreicher und fehlerhafter Tests, Gesamtdauer der Tests. \\
    \multicolumn{2}{p{14.5cm}}{\textit{Wie lange dauert ein kompletter Testdurchlauf? \newline Gibt es Tests, die optimiert werden müssen? \newline Wie oft werden fehlerhafte Tests in die Codebasis aufgenommen?}} \\ \midrule
    \textbf{Code Coverage} & Wie viel Prozent des Quellcodes ist mit Tests abgedeckt. \\
    \multicolumn{2}{p{14.5cm}}{\textit{Gibt es Module, die nicht oder schlecht getestet sind? \newline Wie sieht die Entwicklung der Testabdeckung über die Zeit aus?}} \\ \midrule
    \textbf{Stresstests oder Benchmarking} & Hier kann das Ergebnisse die unterschiedliche Reports sein. \\
    \multicolumn{2}{p{14.5cm}}{\textit{Ist das Produkt auch noch unter Last verwendbar? \newline Wie verändert sich die Leistung über die Zeit?}} \\ \bottomrule
    \end{tabular}
    \caption{Kennzahlen aus den \ac{CI}- und \ac{CD}}-Systemen\label{metrics-table-cicd}
  \end{table}
  
  \begin{table}[H]
    \centering
    \begin{tabular}{p{5cm}p{9.5cm}} \toprule
    \textbf{CPU Nutzung} & Auslastung der Prozessoren über die Zeit. \\
    \textbf{Heap Size} & Auslastung des Heap über die Zeit. \\
    \multicolumn{2}{p{14.5cm}}{\textit{Arbeitet die Software technisch effizient? \newline Ist die Hardware ausreichend? \newline Gibt es eine erhöhte Auslastung nach einer Änderung?}} \\ \midrule
    \textbf{Fehlerraten} & Anzahl Fehler über die Zeit (kann aus dem Logging kommen). \\
    \multicolumn{2}{p{14.5cm}}{\textit{Werden seit einer Änderung mehr Fehler produziert? \newline Wie entwickelt sich die Fehlerrate über die Zeit?}} \\ \midrule
    \textbf{Antwortzeiten} & Dauer der Verarbeitung bestimmter Anfragen. \\
    \multicolumn{2}{p{14.5cm}}{\textit{Reagiert und arbeitet das Produkt noch schnell genug? \newline Gibt es Geschwindigkeitsprobleme seit der letzten Änderung? \newline Wie entwickeln sich die Antwortzeiten über die Zeit?}} \\ \midrule
    \textbf{Benutzeranzahl} & Anzahl gleichzeitiger Benutzer in der Applikation über die Zeit. \\
    \multicolumn{2}{p{14.5cm}}{\textit{Wie entwickeln sich die Nutzerzahlen mit der Zeit? \newline Geht das Produkt in die richtige Richtung? \newline Ist mit höheren Lasten zu rechnen?}} \\ \midrule
    \textbf{Aufenthaltsdauer} & Verweildauer der Benutzer auf bestimmten Seiten. \\
    \multicolumn{2}{p{14.5cm}}{\textit{Welche Features werden besonders oft / selten genutzt? \newline Hat das neue Feature den gewünschten Effekt? Wird es genutzt?}} \\ \midrule
    \textbf{Conversion Rate} & Anzahl Benutzer die zu Kunden wurden. \\
    \multicolumn{2}{p{14.5cm}}{\textit{Wie entwickelt sich die Zahl der zahlenden Neukunden?}} \\ \midrule
    \textbf{Semantisches Logging} & Strukturierte Daten aus dem Logging. \\
    \multicolumn{2}{p{14.5cm}}{\textit{Hier können Daten zu anderen Fragen gesammelt werden, die für den Prozess wichtig sind.}} \\ \bottomrule
    \textbf{Verfügbarkeit} & Verfügbarkeit der Applikation über die Zeit. \\
    \multicolumn{2}{p{14.5cm}}{\textit{Wie hoch ist die Ausfallsicherheit? \newline Wie lange war die Applikation nicht verfügbar?}} \\ \bottomrule
    \end{tabular}
    \caption{Kennzahlen aus den \ac{APM}- und \ac{BI}}-Systemen\label{metrics-table-apm}
  \end{table}

\newpage
\section{Ergebnisse Analyse Retrospektiven}\label{appendix:retros}

\subsection*{Welche guten Entscheidungen haben wir getroffen?}
\begin{enumerate}
    \item sprint (4)
    \item einblick (3)
    \item onboarding (3)
    \item pair (3)
    \item programming (3)
    \item system (3)
    \item arbeit (2)
    \item daily (2)
    \item erledig (2)
    \item information (2)
    \item issu (2)
    \item po (2)
    \item review (2)
    \item reviewing (2)
    \item schnell (2)
    \item stori (2)
    \item urlaub (2)
    \item angenehm (1)
    \item annehm (1)
    \item cloud (1)
    \item dailys (1)
    \item diskussion (1)
    \item dor (1)
    \item durchgefuhrt (1)
    \item einfach (1)
\end{enumerate}

\subsection*{Was haben wir gelernt?}
\begin{enumerate}
    \item sprint (7)
    \item onboarding (4)
    \item team (4)
    \item arbeit (3)
    \item besprech (3)
    \item board (3)
    \item datenfluss (3)
    \item issus (3)
    \item planungswoch (3)
    \item retro (3)
    \item richtlini (3)
    \item system (3)
    \item uberblick (3)
    \item umgestellt (3)
    \item altlast (2)
    \item analogboard (2)
    \item approved (2)
    \item backlog (2)
    \item daily (2)
    \item digital (2)
    \item direkt (2)
    \item genau (2)
    \item impediment (2)
    \item infrastruktur (2)
    \item iso (2)
\end{enumerate}

\subsection*{Was können wir besser machen?}
\begin{enumerate}
    \item sprint (10)
    \item review (7)
    \item checklist (4)
    \item daily (4)
    \item display (4)
    \item doku (4)
    \item issu (4)
    \item po (4)
    \item einarbeitung (3)
    \item einkalkuli (3)
    \item gross (3)
    \item https (3)
    \item java (3)
    \item stori (3)
    \item ablauf (2)
    \item anderung (2)
    \item arbeitspaket (2)
    \item aufnehm (2)
    \item aufteil (2)
    \item backlog (2)
    \item blocked (2)
    \item dokumenti (2)
    \item erledig (2)
    \item geplant (2)
    \item geschatzt (2)
\end{enumerate}

\subsection*{Was nervt uns noch immer?}
\begin{enumerate}
    \item problem (11)
    \item updat (11)
    \item apis (10)
    \item archiv (10)
    \item erreichbar (10)
    \item infrastruktur (10)
    \item jenkin (10)
    \item test (9)
    \item umgebung (9)
    \item dba (8)
    \item eingerichtet (8)
    \item jndi (8)
    \item laut (8)
    \item verwendbar (8)
    \item arbeit (7)
    \item impediment (7)
    \item iso (7)
    \item lang (7)
    \item mitarbeit (6)
    \item mehr (5)
    \item wichtig (5)
    \item anderung (4)
    \item aufteilbar (4)
    \item gross (4)
    \item klar (4)
\end{enumerate}

\newpage
\section{Umfrage Scrum Team}

\subsection{Fragebogen}\label{appendix:questions}
\includepdf[pages=-, scale=0.9, pagecommand={\thispagestyle{plain}}]{appendix/fragebogen.pdf}

\subsection{Ergebnisse}\label{appendix:answers}
\includepdf[pages=-, scale=0.9, pagecommand={\thispagestyle{plain}}]{appendix/fragebogen-ergebnis.pdf}

\newpage
\section{Transkripte Interviews}\label{appendix:transcript}

Im Folgenden finden sich die Transkripte zu den Interviews, die zu Evaluationszwecken geführt wurden.
Der Interviewer wird mit A und der Interviewte mit B abgekürzt.

\subsection{Product Owner}

A\@: ``Frage'' \\
B\@: ``Antwort'' \\
A\@: ``Frage'' \\
B\@: ``Antwort''

\subsection{Scrum Master}

A\@: ``Frage'' \\
B\@: ``Antwort'' \\
A\@: ``Frage'' \\
B\@: ``Antwort''

\subsection{Developer}

A\@: ``Frage'' \\
B\@: ``Antwort'' \\
A\@: ``Frage'' \\
B\@: ``Antwort''


\chapter*{Eidesstattliche Erklärung}
\addcontentsline{toc}{chapter}{Eidesstattliche Erklärung}
Ich erkläre hiermit an Eides statt, dass ich die vorliegende Masterarbeit selbstständig und ohne Benutzung anderer als der angegebenen Hilfsmittel angefertigt habe. Die aus fremden Quellen direkt oder indirekt übernommenen Stellen sind als solche kenntlich gemacht. Die Arbeit wurde bisher weder in gleicher noch in ähnlicher Form einer anderen Prüfungsbehörde vorgelegt und auch noch nicht veröffentlicht.

\vspace{3cm}
\noindent
Dornbirn, am [Tag. Monat Jahr anführen]\hfill [Vor- und Nachname Verfasser/in]


\end{document}
